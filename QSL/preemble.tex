

%\usepackage{graphicx}% Include figure files
\usepackage{dcolumn,bm,comment}% Align table columns on decimal point
\usepackage{subfiles}
% 数式
\usepackage{amsmath,amsfonts,bm,mathcomp,physics, amssymb}
\numberwithin{equation}{section}

% 画像
\usepackage{color,float,subfigure}
\usepackage[dvipdfmx]{graphicx}
% TikZ
\usepackage{tikz}
\usetikzlibrary{intersections, calc, arrows.meta}
% 囲み
\usepackage{ascmac,fancybox}
\usepackage{tcolorbox}
\tcbuselibrary{breakable, skins, theorems}

\usepackage{listings,jvlisting} %日本語のコメントアウトをする場合jvlisting(もしくはjlisting)が必要
\usepackage[margin=30truemm]{geometry}

%\usepackage[x11names]{xcolor}
\usepackage[dvipsnames]{xcolor}
%\usepackage[colorlinks,citecolor=SpringGreen4,linkcolor=purple,dvipdfmx]{hyperref} 
\usepackage[colorlinks,citecolor=OliveGreen,linkcolor=BrickRed,dvipdfmx]{hyperref} 
\usepackage{pxjahyper}
%\usepackage{hyperref}

\usepackage[backend=bibtex,style=phys,%
articletitle=false,biblabel=brackets,%
chaptertitle=false,pageranges=false]{biblatex}
\addbibresource{reference.bib}

\newtheorem{theorem}{定理}
\newtheorem{definition}{定義}
%\newtheorem{mybox}{box}
\newtcbtheorem{mybox}{My Theorem}{enhanced, %TikZの内部処理を導入する.ある程度複雑なものには必須.
    colback = white,
    colframe = green!35!black,
    fonttitle = \bfseries,
    breakable = true,
}{hoge}