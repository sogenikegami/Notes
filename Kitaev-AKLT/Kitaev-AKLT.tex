\documentclass[11pt, aps, longbibliography]{article}


%\usepackage{graphicx}% Include figure files
\usepackage{dcolumn,bm,comment}% Align table columns on decimal point
\usepackage{subfiles}
% 数式
\usepackage{amsmath,amsfonts,bm,mathcomp,physics, amssymb}
\numberwithin{equation}{section}

% 画像
\usepackage{color,float,subfigure}
\usepackage[dvipdfmx]{graphicx}
% TikZ
\usepackage{tikz}
\usetikzlibrary{intersections, calc, arrows.meta}
% 囲み
\usepackage{ascmac,fancybox}
\usepackage{tcolorbox}
\tcbuselibrary{breakable, skins, theorems}

\usepackage{listings,jvlisting} %日本語のコメントアウトをする場合jvlisting(もしくはjlisting)が必要
\usepackage[margin=30truemm]{geometry}

%\usepackage[x11names]{xcolor}
\usepackage[dvipsnames]{xcolor}
%\usepackage[colorlinks,citecolor=SpringGreen4,linkcolor=purple,dvipdfmx]{hyperref} 
\usepackage[colorlinks,citecolor=OliveGreen,linkcolor=BrickRed,dvipdfmx]{hyperref} 
\usepackage{pxjahyper}
%\usepackage{hyperref}

\usepackage[backend=bibtex,style=phys,%
articletitle=false,biblabel=brackets,%
chaptertitle=false,pageranges=false]{biblatex}
\addbibresource{reference.bib}

\newtheorem{theorem}{定理}
\newtheorem{definition}{定義}
%\newtheorem{mybox}{box}
\newtcbtheorem{mybox}{My Theorem}{enhanced, %TikZの内部処理を導入する.ある程度複雑なものには必須.
    colback = white,
    colframe = green!35!black,
    fonttitle = \bfseries,
    breakable = true,
}{hoge}
\addbibresource{reference}
\begin{document}
\title{Note: Kitaev+AKLT model}
\author{Sogen Ikegami}
\date{\today}
\maketitle

\tableofcontents

\section*{preface}

This note aims to summarize and discuss the ambiguity of bi-quadrutic and bi-cubic interaction when we consider SU$(N)$ spin coherent state.
Here we skip the definition and properties of SU$(N)$ coherent state. When we want to check them, Refs. \cite{Nemoto_2000,PhysRevB.108.L241108} would be helpful.

\newpage
\section{Setup}
    \subsection{Model}
        Now we are considering the $S=3/2$ system on the honeycomb lattice with following Hamiltonian:
        \begin{equation}\label{eq:Hamiltonian}
            H = \cos (2\pi \xi) H_{\rm AKLT} + \sin (2\pi \xi) H_{\rm Kitaev},
        \end{equation}
        with
        \begin{align}
            &H_{\rm AKLT} = \sum_{\langle i,j \rangle}J_1(\hat{\bold{S}}_{i}\cdot \hat{\bold{S}}_{j}) + J_2(\hat{\bold{S}}_{i}\cdot \hat{\bold{S}}_{j})^2 + J_3(\hat{\bold{S}}_{i}\cdot \hat{\bold{S}}_{j})^3, \\
            &H_{\rm Kitaev} = \sum_{\alpha \in \{x,y,z\}}\sum_{\langle i,j \rangle_\alpha} K_{\alpha}\hat{S}_{i}^{\alpha}\hat{S}_{j}^{\alpha}.
        \end{align}
        $J_1=1, J_2=116/243, J_3=16/243$ recovers the original 2D AKLT Hamiltonian on the honeycomb lattice. In my calculation, I set $K_x=K_y=K_z=1$ for simplicity.

    \subsection{Spin operators}
        Here we confirm the explicit spin operators in matrix forms. It is assumed to be in $S_z$ basis from now on, if no mention is made.

        \begin{itemize}
            \item $S=1/2$
        \end{itemize}
        \begin{equation}
            \hat{S}_x = \frac{1}{2}\sigma_x = \frac{1}{2}\begin{pmatrix} 0 & 1 \\ 1 & 0 \end{pmatrix}, \ 
            \hat{S}_y = \frac{1}{2}\sigma_y = \frac{1}{2}\begin{pmatrix} 0 & -i \\ i & 0 \end{pmatrix}, \ 
            \hat{S}_z = \frac{1}{2}\sigma_y = \frac{1}{2}\begin{pmatrix} 1 & 0 \\ 0 & -1 \end{pmatrix}.
        \end{equation}
        They satisfy the following relation, which is unique in $S=1/2$,
        \begin{equation}\label{eq:pauli}
            \hat{S}_{i}\hat{S}_j = \frac{i}{2}\epsilon_{ijk}\hat{S}_k + \frac{\bold{1}_{2\times2}}{4}\delta_{ij}.
        \end{equation}

        \begin{itemize}
            \item $S=1$
        \end{itemize}
        \begin{equation}\label{eq:SU3-matrix}
            \hat{S}_x = \frac{1}{\sqrt{2}}\begin{pmatrix} 0 & 1 & 0 \\ 1 & 0 & 1 \\ 0 & 1 & 0 \end{pmatrix}, \ 
            \hat{S}_y = \frac{1}{\sqrt{2}i}\begin{pmatrix} 0 & 1 & 0 \\ -1 & 0 & 1 \\ 0 & -1 & 0 \end{pmatrix}, \ 
            \hat{S}_z = \begin{pmatrix} 1 & 0 & 0 \\ 0 & 0 & 0 \\ 0 & 0 & -1 \end{pmatrix}.
        \end{equation}
        Eq.\eqref{eq:pauli} no longer holds\footnote{Commutation relationship of Lie algebra $\mathfrak{su}$(2) \begin{equation}\label{eq:commutation-SU2}
            \left[S^i, S^j\right] = i\epsilon_{ijk}S^k
        \end{equation}
        holds also for matrices \eqref{eq:SU3-matrix} and \eqref{eq:SU4-matrix}. Matrices \eqref{eq:SU3-matrix} and \eqref{eq:SU4-matrix} form SU(2) subalgebra in each algebra. In papers they are said to be higher dimensional representation of SU(2),
        because commutation \eqref{eq:commutation-SU2} is special commutation relationship for $\mathfrak{su}$(2), while commutation in general $\mathfrak{su}$(N) (general Lie algebra) is \begin{equation}
            \left[S^i, S^j\right] = if_{ijk}S^k,
        \end{equation}
        where $f$ is structure constant.
        }
        . For example,
        \begin{equation}
            \hat{S}_x\hat{S}_y = \frac{1}{2i} \begin{pmatrix}-1 & 0 & 1 \\ 0 & 0 & 0 \\ -1 & 0 & 1\end{pmatrix} \neq \hat{S}_z.
        \end{equation}

        \begin{itemize}
            \item $S=3/2$
        \end{itemize}
        \begin{equation}\label{eq:SU4-matrix}
            \hat{S}_x = \frac{1}{2}\begin{pmatrix} 0 & \sqrt{3} & 0 & 0 \\ \sqrt{3} & 0 & 2 & 0 \\ 0 & 2 & 0 & \sqrt{3} \\ 0 & 0 & \sqrt{3} & 0 \end{pmatrix}, \ 
            \hat{S}_y = \frac{1}{2i}\begin{pmatrix} 0 & \sqrt{3} & 0 & 0 \\ -\sqrt{3} & 0 & 2 & 0 \\ 0 & -2 & 0 & \sqrt{3} \\ 0 & 0 & -\sqrt{3} & 0 \end{pmatrix}, \ 
            \hat{S}_z = \begin{pmatrix} \frac{3}{2} & 0 & 0 & 0 \\ 0 & \frac{1}{2} & 0 & 0 \\ 0 & 0 & -\frac{1}{2} & 0 \\ 0 & 0 & 0 & -\frac{3}{2} \end{pmatrix}.
        \end{equation}

    \subsection{Multipole Operators}
        Higher spins ($S>1/2$) possess local multipolar degrees of freedom, such as quadrupoles and octupoles.
        Multipole operators are defined by the product of the generators of the $\mathfrak{su}(2)$ subalgebra of $\mathfrak{su}(N)$.
        The identity matrix, dipole (usual $S_x, S_y, S_z$) operators and multipole operators form the generators of the Lie algebra $\mathfrak{su}(N)$.

        When $S\geq 1$, the local Hilbert space has quadrupole degrees of freedom. 
        Quadrupole has 5 components and their operators are usually defined as follows.
        \begin{align}\label{eq:defQ}
            Q_{3z^2-x^2-y^2} &= \frac{1}{\sqrt{3}}\left( 2\hat{S}_z^2 - \hat{S}_x^2 - \hat{S}_y^2, \right) \notag \\
            Q_{x^2-y^2} &= \hat{S}_x^2 - \hat{S}_y^2, \notag \\
            Q_{xy} &= \hat{S}_x\hat{S}_y + \hat{S}_y\hat{S}_x, \\
            Q_{yz} &= \hat{S}_y\hat{S}_z + \hat{S}_z\hat{S}_y, \notag \\
            Q_{zx} &= \hat{S}_z\hat{S}_x + \hat{S}_x\hat{S}_z. \notag 
        \end{align}
        It should be noted that the overall coefficient is arbitrary when we only consider quadrupoles. In contrast, the ratio of the coefficients between each component must not be changed,
        and the overall coefficient is also important if we deal with the quadrupole together with other moments.
        In such a situation, it may be mathematically standard to impose the orthonormal condition for each Lie algebra generators
        \begin{equation}
            \Tr (T_\alpha^\dagger T_\beta) = 5\delta_{\alpha \beta} = \Tr (\hat{S}_{z}^\dagger \hat{S}_{z}).
        \end{equation}
        $5$ in the middle is set so that the dipole operators are normalized as usual definition of spin operators in eq.\eqref{eq:SU4-matrix}\footnote{In mathematical context, this normalization factor is usually taken as 1. This factor affects only the overall prefactor of the Hamiltonian in my case.}.
        Note that this constant depends on the spin-$S$.
        This constraint modifies the definition of the quadrupole operators as
        \begin{align}\label{eq:defQ2}
            Q_{3z^2-x^2-y^2} &= \frac{\sqrt{5}}{6}\left( 2\hat{S}_z^2 - \hat{S}_x^2 - \hat{S}_y^2, \right) \notag \\
            Q_{x^2-y^2} &= \frac{1}{2}\sqrt{\frac{5}{3}} \left( \hat{S}_x^2 - \hat{S}_y^2, \right) \notag \\
            Q_{xy} &= \frac{1}{2}\sqrt{\frac{5}{3}} \left( \hat{S}_x\hat{S}_y + \hat{S}_y\hat{S}_x\right), \\
            Q_{yz} &= \frac{1}{2}\sqrt{\frac{5}{3}} \left( \hat{S}_y\hat{S}_z + \hat{S}_z\hat{S}_y\right), \notag \\
            Q_{zx} &= \frac{1}{2}\sqrt{\frac{5}{3}} \left( \hat{S}_z\hat{S}_x + \hat{S}_x\hat{S}_z\right). \notag 
        \end{align}
        These multipole operators can be obtained by symmetrizing each term in the spherical harmonics with appropriate modification of the coefficients.
        For example, the $l=2$ parts of the linear combination of the spherical harmonics so that the combined one is real are
        \begin{align}\label{eq:spheQ}
            Y_{2,-2} &= i\frac{1}{\sqrt{2}}\left( Y_2^{-2} - Y_2^2 \right) = \frac{1}{2}\sqrt{\frac{15}{\pi}}\frac{xy}{r^2} \propto \sqrt{3}xy, \notag \\
            Y_{2,-1} &= i\frac{1}{\sqrt{2}}\left( Y_2^{-1} + Y_2^1 \right) = \frac{1}{2}\sqrt{\frac{15}{\pi}}\frac{yz}{r^2} \propto \sqrt{3}yz, \notag \\
            Y_{2,0} &= Y_2^0 = \frac{1}{4}\sqrt{\frac{5}{\pi}}\frac{2z^2-x^2-y^2}{r^2} \propto \frac{1}{2}(2z^2-x^2-y^2), \\
            Y_{2,1} &= \frac{1}{\sqrt{2}}\left( Y_2^{-1} - Y_2^1 \right) = \frac{1}{2}\sqrt{\frac{15}{\pi}}\frac{zx}{r^2} \propto \sqrt{3}zx, \notag \\
            Y_{2,2} &= \frac{1}{\sqrt{2}}\left( Y_2^{-2} + Y_2^2 \right) = \frac{1}{4}\sqrt{\frac{15}{\pi}}\frac{x^2-y^2}{r^2} \propto \frac{\sqrt{3}}{2}(x^2-y^2). \notag 
        \end{align}
        Symmetrizing the $xy$ term in the first line yields $\frac{1}{2}(S_xS_y+S_yS_x)$. Multiplying a constant to all components leads to the definition \eqref{eq:defQ}\eqref{eq:defQ2}.
        We get the octupole operators at the same manner as quadrupole operators. In the case of the quadrupole operators, eq.\eqref{eq:spheQ} is the most natural representation of the spherical harmonics. 
        However, as for octupoles, there are several representations of the spherical harmonics such as spherical(tesseral) harmonics and cubic harmonics.
        Both are the linear combinations of spherical harmonics, but the 
        tesseral harmonics are the real part of the spherical harmonics, and the cubic harmonics are made so that they form the irreducible representation of the cubic point group $O_h$.
        
        Tesseral (spherical) based octupole operators are
        \begin{align}\label{eq:sphe}
            Z_3^0 = Y_3^0, \  &O_{z^3} = \frac{1}{3}\left( 2\hat{S}_z^3 - \left( \hat{S}_z\hat{S}_x^2 + \hat{S}_x\hat{S}_z\hat{S}_x + \hat{S}_x^2\hat{S}_z \right) - \left( \hat{S}_z\hat{S}_y^2 + \hat{S}_y\hat{S}_z\hat{S}_y + \hat{S}_y^2\hat{S}_z \right)  \right), \notag \\
            Z_3^1 = Y_3^{-1} - Y_3^1, \  &O_{xz^2} = \frac{1}{3\sqrt{6}}\left( 4\left( \hat{S}_x\hat{S}_z^2 + \hat{S}_z\hat{S}_x\hat{S}_z + \hat{S}_z^2\hat{S}_x \right) -\left( \hat{S}_x\hat{S}_y^2 + \hat{S}_y\hat{S}_x\hat{S}_y + \hat{S}_y^2\hat{S}_x \right) -3\hat{S}_x^3 \right), \notag \\
            Z_3^{-1} = Y_3^{-1} + Y_3^1, \  &O_{yz^2} = \frac{1}{3\sqrt{6}}\left( 4\left( \hat{S}_y\hat{S}_z^2 + \hat{S}_z\hat{S}_y\hat{S}_z + \hat{S}_z^2\hat{S}_y \right) -\left( \hat{S}_y\hat{S}_x^2 + \hat{S}_x\hat{S}_y\hat{S}_x + \hat{S}_x^2\hat{S}_y \right) -3\hat{S}_y^3 \right), \notag \\
            Z_3^{-2} = Y_3^{-2} - Y_3^2, \  &O_{xyz} = \frac{1}{3}\sqrt{\frac{5}{3}} \left(\hat{S}_x\hat{S}_y\hat{S}_z + \hat{S}_x\hat{S}_z\hat{S}_y + \hat{S}_y\hat{S}_x\hat{S}_z + \hat{S}_y\hat{S}_z\hat{S}_x + \hat{S}_z\hat{S}_x\hat{S}_y + \hat{S}_z\hat{S}_y\hat{S}_x \right), \\
            Z_3^{2} = Y_3^{-2} + Y_3^2, \  &O_{z(x^2-y^2)} = \frac{1}{3}\sqrt{\frac{5}{3}} \left( \left( \hat{S}_z\hat{S}_x^2 + \hat{S}_x\hat{S}_z\hat{S}_x + \hat{S}_x^2\hat{S}_z \right) - \left( \hat{S}_z\hat{S}_y^2 + \hat{S}_y\hat{S}_z\hat{S}_y + \hat{S}_y^2\hat{S}_z \right) \right), \notag \\
            Z_3^{3} = Y_3^{-3} - Y_3^3, \  &O_{x(x^2-3y^2)} = \frac{1}{3}\sqrt{\frac{5}{2}} \left( \hat{S}_x^3 -\left( \hat{S}_x\hat{S}_y^2 + \hat{S}_y\hat{S}_x\hat{S}_y + \hat{S}_y^2\hat{S}_x \right) \right), \notag \\
            Z_3^{-3} = Y_3^{-3} + Y_3^3, \  &O_{y(3x^2-y^2)} = \frac{1}{3}\sqrt{\frac{5}{2}} \left( \left( \hat{S}_y\hat{S}_x^2 + \hat{S}_x\hat{S}_y\hat{S}_x + \hat{S}_x^2\hat{S}_y \right) - \hat{S}_y^3 \right). \notag 
        \end{align}
        The cubic representation based octupole operators are
        \begin{align}
            O_{A_{2u}} &= \frac{1}{3}\sqrt{\frac{5}{3}} \left( \hat{S}_x\hat{S}_y\hat{S}_z + \hat{S}_x\hat{S}_z\hat{S}_y + \hat{S}_y\hat{S}_x\hat{S}_z + \hat{S}_y\hat{S}_z\hat{S}_x + \hat{S}_z\hat{S}_x\hat{S}_y + \hat{S}_z\hat{S}_y\hat{S}_x \right), \notag \\
            O_x^\alpha &= \frac{1}{3} \left( 2\hat{S}_x^3 -\left(\hat{S}_x\hat{S}_y^2+\hat{S}_y\hat{S}_x\hat{S}_y + \hat{S}_y^2\hat{S}_x\right) - \left(\hat{S}_x\hat{S}_z^2 + \hat{S}_z\hat{S}_x\hat{S}_z + \hat{S}_z^2\hat{S}_x\right) \right), \notag \\
            O_y^\alpha &= \frac{1}{3} \left( 2\hat{S}_y^3 -\left(\hat{S}_y\hat{S}_z^2+\hat{S}_z\hat{S}_y\hat{S}_z + \hat{S}_z^2\hat{S}_y\right) - \left(\hat{S}_y\hat{S}_x^2 + \hat{S}_x\hat{S}_y\hat{S}_x + \hat{S}_x^2\hat{S}_y\right) \right), \notag \\
            O_z^\alpha &= \frac{1}{3} \left( 2\hat{S}_z^3 -\left(\hat{S}_z\hat{S}_x^2+\hat{S}_x\hat{S}_z\hat{S}_x + \hat{S}_x^2\hat{S}_z\right) - \left(\hat{S}_z\hat{S}_y^2 + \hat{S}_y\hat{S}_z\hat{S}_y + \hat{S}_y^2\hat{S}_z\right) \right), \\
            O_x^\beta &= \frac{1}{3}\sqrt{\frac{5}{3}} \left( \left(\hat{S}_x\hat{S}_y^2+\hat{S}_y\hat{S}_x\hat{S}_y + \hat{S}_y^2\hat{S}_x\right) - \left(\hat{S}_x\hat{S}_z^2 + \hat{S}_z\hat{S}_x\hat{S}_z + \hat{S}_z^2\hat{S}_x\right) \right), \notag \\
            O_y^\beta &= \frac{1}{3}\sqrt{\frac{5}{3}} \left(\left(\hat{S}_y\hat{S}_z^2+\hat{S}_z\hat{S}_y\hat{S}_z + \hat{S}_z^2\hat{S}_y\right) - \left(\hat{S}_y\hat{S}_x^2 + \hat{S}_x\hat{S}_y\hat{S}_x + \hat{S}_x^2\hat{S}_y\right) \right), \notag \\
            O_z^\beta &= \frac{1}{3}\sqrt{\frac{5}{3}} \left(\left(\hat{S}_z\hat{S}_x^2+\hat{S}_x\hat{S}_z\hat{S}_x + \hat{S}_x^2\hat{S}_z\right) - \left(\hat{S}_z\hat{S}_y^2 + \hat{S}_y\hat{S}_z\hat{S}_y + \hat{S}_y^2\hat{S}_z\right) \right). \notag 
        \end{align}
        $O^\alpha$ are $T_{1u}$ channel and $O^\beta$ in $T_{2u}$.

        We can rewrite our Hamiltonian by $\mathfrak{su}(4)$ generators. $\mathbf{S}_i \cdot \mathbf{S}_j, \mathbf{Q}_i \cdot \mathbf{Q}_j, \mathbf{O}_i \cdot \mathbf{O}_j$ are now $4^2=16$ dimensional matrix. 
        Comparing the matrix elements of these matrices and $(\mathbf{S}_i \cdot \mathbf{S}_j)^2, (\mathbf{S}_i \cdot \mathbf{S}_j)^3$, we can see the following identity
        \begin{align}
            \frac{6}{5}(\mathbf{Q}_i \cdot \mathbf{Q}_j) &= (\mathbf{S}_i \cdot \mathbf{S}_j)^2 + \frac{1}{2}(\mathbf{S}_i \cdot \mathbf{S}_j) - \frac{75}{16}, \\
            \frac{9}{10}(\mathbf{O}_i \cdot \mathbf{O}_j) &= (\mathbf{S}_i \cdot \mathbf{S}_j)^3 + 2(\mathbf{S}_i \cdot \mathbf{S}_j)^2 - \frac{507}{80}(\mathbf{S}_i \cdot \mathbf{S}_j) - \frac{225}{32},
        \end{align}
        hence
        \begin{align}
            (\mathbf{S}_i \cdot \mathbf{S}_j)^2 &= \frac{6}{5}(\mathbf{Q}_i \cdot \mathbf{Q}_j) - \frac{1}{2}(\mathbf{S}_i \cdot \mathbf{S}_j) + \frac{75}{16}, \\
            (\mathbf{S}_i \cdot \mathbf{S}_j)^3 &= \frac{9}{10}(\mathbf{O}_i \cdot \mathbf{O}_j) - 2\left( \frac{6}{5}(\mathbf{Q}_i \cdot \mathbf{Q}_j) - \frac{1}{2}(\mathbf{S}_i \cdot \mathbf{S}_j) + \frac{75}{16} \right) + \frac{507}{80}(\mathbf{S}_i \cdot \mathbf{S}_j) + \frac{225}{32} \notag \\
            &= \frac{9}{10}(\mathbf{O}_i \cdot \mathbf{O}_j) - \frac{12}{5}(\mathbf{Q}_i \cdot \mathbf{Q}_j) + \frac{587}{80}(\mathbf{S}_i \cdot \mathbf{S}_j) - \frac{75}{32}.
        \end{align}

\newpage

\section{Ambiguities for Classical Model}

\subsection{SU(2)}\label{sec:SU2}
Here we see what happens when we apply SU(2) coherent state to our Hamiltonian \eqref{eq:Hamiltonian}.
First, Kitaev and Heisenberg (Bi-linear) interaction do not pose any problem. 
How about bi-quadrutic and bi-cubic terms? In \textbf{\emph{Operator level}},
\begin{align}
    (\hat{\bold{S}}_i \cdot \hat{\bold{S}}_j)^2 &= (\hat{S}_i^x\hat{S}_j^x + \hat{S}_i^y\hat{S}_j^y + \hat{S}_i^z\hat{S}_j^z) (\hat{S}_i^x\hat{S}_j^x + \hat{S}_i^y\hat{S}_j^y + \hat{S}_i^z\hat{S}_j^z) \\
    &= (\hat{S}_i^x)^2 (\hat{S}_j^x)^2 + (\hat{S}_i^y)^2 (\hat{S}_j^y)^2 + (\hat{S}_i^z)^2 (\hat{S}_j^z)^2 \\
    & \quad + (\hat{S}_i^x \hat{S}_i^y) (\hat{S}_j^x \hat{S}_j^y) + (\hat{S}_i^y \hat{S}_i^x) (\hat{S}_j^y \hat{S}_j^x) + (\text{cyc.}) \\
    &= \left( \frac{1}{4} \right)^2 \times 3 \\
    & \quad +\left(\frac{i}{2}\hat{S}_i^z\right)\left(\frac{i}{2}\hat{S}_j^z\right) + \left(-\frac{i}{2}\hat{S}_i^z\right)\left(-\frac{i}{2}\hat{S}_j^z\right) + (\text{cyc.}) \\
    &= \frac{3}{16} - \frac{1}{2}(\hat{S}_i^x\hat{S}_j^x + \hat{S}_i^y\hat{S}_j^y + \hat{S}_i^z\hat{S}_j^z) \\
    &= \frac{3}{16} - \frac{1}{2}\hat{\bold{S}}_i \cdot \hat{\bold{S}}_j, \label{eq:SU2-quad1}
\end{align}

\begin{align}
    (\hat{\bold{S}}_i \cdot \hat{\bold{S}}_j)^3  &= \left(\frac{3}{16} - \frac{1}{2}\hat{\bold{S}}_i \cdot \hat{\bold{S}}_j \right) (\hat{\bold{S}}_i \cdot \hat{\bold{S}}_j ) \\
    &= \frac{3}{16}(\hat{\bold{S}}_i \cdot \hat{\bold{S}}_j ) - \frac{1}{2} \left( \frac{3}{16} - \frac{1}{2}\hat{\bold{S}}_i \cdot \hat{\bold{S}}_j \right) \\
    &= -\frac{3}{32} + \frac{7}{16} (\hat{\bold{S}}_i \cdot \hat{\bold{S}}_j). \label{eq:SU2-cubic1}
\end{align}
Hence, \textbf{\emph{Bi-quadrutic, Bi-cubic terms are transformed to Bi-linear term in operator level in SU(2) formalism.}}
This arises because Pauli matrices cannot express quadrupoles and octupoles.
In my calculation, I treated these bi-quadrutic and bi-cubic terms as \eqref{eq:SU2-quad1} and \eqref{eq:SU2-cubic1}.
Althought this formalism is straight forward and quite correct in operator level, as Rico San pointed out, this approach might \textbf{\emph{not be appropriate as a classical limit of AKLT model}}. 

Rico San's suggestion is as follows.
Our Hilbert space is limited to $\mathbb{C}P^1$, we have to give up something. 
Eqs.\eqref{eq:SU2-quad1} and \eqref{eq:SU2-cubic1} is analytically correct but give up physical meanings (“freeze out” the quadrupoles and octupoles).
Instead, we give up the analytical correctness and recover the physical meanings, i.e., recovering AKLT model considered for $\mathbb{C}P^1$ subspaces by
\begin{align}
    (\hat{\bold{S}}_i \cdot \hat{\bold{S}}_j)^2 &\rightarrow \langle \hat{\bold{S}}_i \cdot \hat{\bold{S}}_j \rangle \langle \hat{\bold{S}}_i \cdot \hat{\bold{S}}_j \rangle,  \label{eq:SU2-quad2} \\
    (\hat{\bold{S}}_i \cdot \hat{\bold{S}}_j)^3 &\rightarrow \langle \hat{\bold{S}}_i \cdot \hat{\bold{S}}_j \rangle^3.\label{eq:SU2-cubic2}
\end{align}


\begin{tcolorbox}
    \textbf{Ambiguity-1}
    What is the appropriate classical limit of our model?
    \begin{enumerate}
        \item Eqs.\eqref{eq:SU2-quad1} and \eqref{eq:SU2-cubic1}, used in my calculation. Correct at analytical level, but seems not appropriate \emph{as the classical limit of AKLT model}.
        \item Eqs.\eqref{eq:SU2-quad2} and \eqref{eq:SU2-cubic2}, suggested by Rico San. It discards the operator level correctness, but captures quadrutic and cubic terms.
    \end{enumerate}
\end{tcolorbox}

2nd approach seems more appropriate as the classical limit in my view. Fortunately or not, since we have done the 1st approach, comparing two approaches' results would be interesting.

\subsection{SU(3)}\label{sec:SU3}
In SU(2) formalism, bi-quadrutic and bi-cubic terms posed the problems. The same situation arises in SU(3) formalism by bi-cubic term.
I do not show all (27) combinations of three spin-1 operators, but the simpler examples are
\begin{align}
    &(\hat{S}_i^x)^3 = \frac{1}{2\sqrt{2}} \begin{pmatrix} 1 & 0 & 1 \\ 0 & 2 & 0 \\ 1 & 0 & 1\end{pmatrix}\begin{pmatrix} 0 & 1 & 0 \\ 1 & 0 & 1 \\ 0 & 1 & 0\end{pmatrix} =\frac{1}{2\sqrt{2}} \begin{pmatrix} 0 & 2 & 0 \\ 2 & 0 & 2 \\ 0 & 2 & 0  \end{pmatrix} = S_i^x, \\
    &(\hat{S}_i^y)^3 = -\frac{1}{2\sqrt{2}i} \begin{pmatrix} -1 & 0 & 1 \\ 0 & -2 & 0 \\ 1 & 0 & -1\end{pmatrix}\begin{pmatrix} 0 & 1 & 0 \\ -1 & 0 & 1 \\ 0 & -1 & 0\end{pmatrix} = -\frac{1}{2\sqrt{2}} \begin{pmatrix} 0 & -2 & 0 \\ 2 & 0 & -2 \\ 0 & 2 & 0  \end{pmatrix} = S_i^y, \\
    &(\hat{S}_i^z)^3 = S_i^z.
\end{align}
So the same ambiguity as SU(2) (ambiguity-1) arises. As Rico San pointed out, the 2nd approach in ambiguity-1 poses another ambiguity. Let's see it.
As we did in SU(2), let's try to replace the bi-cubic term into products of some expectation values.

\begin{align}
    (\hat{\bold{S}}_i \cdot \hat{\bold{S}}_j)^3 &\rightarrow \langle \hat{\bold{S}}_i \cdot \hat{\bold{S}}_j \rangle^3,  \label{eq:SU3-1} \\
    (\hat{\bold{S}}_i \cdot \hat{\bold{S}}_j)^3 &\rightarrow \langle \hat{(\bold{S}}_i \cdot \hat{\bold{S}}_j)^2 \rangle \langle \hat{\bold{S}}_i \cdot \hat{\bold{S}}_j \rangle,  \label{eq:SU3-2} \\
    (\hat{\bold{S}}_i \cdot \hat{\bold{S}}_j)^3 &\rightarrow \langle \hat{\bold{S}}_i \cdot \hat{\bold{S}}_j \rangle \langle \hat{(\bold{S}}_i \cdot \hat{\bold{S}}_j)^2 \rangle.  \label{eq:SU3-3} \\
\end{align}
Eq.\eqref{eq:SU3-1} uses only dipoles. Now we have dipole and quadrupoles, so eq.\eqref{eq:SU3-1} might be inappropriate.
Eqs.\eqref{eq:SU3-2} and \eqref{eq:SU3-3} were suggested by Rico San as two independent decompositions, but I think these are equivalent, 
because expectation values are commutative.
Anyway, as we see, there are several ways to decompose the cubic term into the product of $S^2$ and $S$.
Another decomposition can be constructed. Now we have dipoles and quadrupoles. Bi-quadrutic term can be decomposed into quadrupoles and dipoles as follows.
\begin{equation}
    (\hat{\bold{S}}_i \cdot \hat{\bold{S}}_j)^2 = a(\hat{\bold{Q}}_i \cdot \hat{\bold{Q}}_j) + b(\hat{\bold{S}}_i \cdot \hat{\bold{S}}_j) + c. \label{eq:S2-decomposition}
\end{equation}
The coefficients are dependent of the definition of quadrupole operators, but it does not matter here.
Another decomposition is achieved by using this decomposition as:
\begin{align}
    (\hat{\bold{S}}_i \cdot \hat{\bold{S}}_j)^3 &= (\hat{\bold{S}}_i \cdot \hat{\bold{S}}_j)^2 (\hat{\bold{S}}_i \cdot \hat{\bold{S}}_j) \\
    &= \left( a(\hat{\bold{Q}}_i \cdot \hat{\bold{Q}}_j) + b(\hat{\bold{S}}_i \cdot \hat{\bold{S}}_j) + c \right)(\hat{\bold{S}}_i \cdot \hat{\bold{S}}_j) \\
    &\rightarrow a\langle(\hat{\bold{Q}}_i \cdot \hat{\bold{Q}}_j)\rangle\langle(\hat{\bold{S}}_i \cdot \hat{\bold{S}}_j)\rangle + ab\langle (\hat{\bold{Q}}_i \cdot \hat{\bold{Q}}_j) \rangle + (b^2+c) \langle (\hat{\bold{S}}_i \cdot \hat{\bold{S}}_j) \rangle + bc. \label{eq:SU3-4}
\end{align}
At the same mannar as eq.\eqref{eq:S2-decomposition}, $(\hat{\bold{S}}_i \cdot \hat{\bold{S}}_j)^3$ can be decomposed into 
\begin{equation}
    (\hat{\bold{S}}_i \cdot \hat{\bold{S}}_j)^3 = \alpha(\hat{\bold{O}}_i \cdot \hat{\bold{O}}_j) + \beta(\hat{\bold{Q}}_i \cdot \hat{\bold{Q}}_j) + \gamma(\hat{\bold{S}}_i \cdot \hat{\bold{S}}_j) + \delta .\label{eq:S2-decomposition}
\end{equation}
In that sense, decomposition eq.\eqref{eq:SU3-4} naturally "translate" the $(\hat{\bold{O}}_i \cdot \hat{\bold{O}}_j)$ term into the product of $(\hat{\bold{S}}_i \cdot \hat{\bold{S}}_j)$ and $(\hat{\bold{Q}}_i \cdot \hat{\bold{Q}}_j)$.

The second ambiguity is summarized as follows.
\begin{tcolorbox}
    \textbf{Ambiguity-2} 
    
    When we take the 2nd approach of ambiguity-1, what is the appropriate decomposition? 
    In other words, what is the most natural classical limit (SU(3) limit) of AKLT model?
    \begin{enumerate}
        \item Eq.\eqref{eq:SU3-1}: $(\hat{\bold{S}}_i \cdot \hat{\bold{S}}_j)^3 \rightarrow \langle \hat{\bold{S}}_i \cdot \hat{\bold{S}}_j \rangle^3,$
        \item Eq.\eqref{eq:SU3-2}: $(\hat{\bold{S}}_i \cdot \hat{\bold{S}}_j)^3 \rightarrow \langle \hat{(\bold{S}}_i \cdot \hat{\bold{S}}_j)^2 \rangle \langle \hat{\bold{S}}_i \cdot \hat{\bold{S}}_j \rangle,$
        \item Eq.\eqref{eq:SU3-4}: $(\hat{\bold{S}}_i \cdot \hat{\bold{S}}_j)^3 \rightarrow a\langle(\hat{\bold{Q}}_i \cdot \hat{\bold{Q}}_j)\rangle\langle(\hat{\bold{S}}_i \cdot \hat{\bold{S}}_j)\rangle + ab\langle (\hat{\bold{Q}}_i \cdot \hat{\bold{Q}}_j) \rangle + (b^2+c) \langle (\hat{\bold{S}}_i \cdot \hat{\bold{S}}_j) \rangle + bc.$
        \item Other decompositions.
    \end{enumerate}
\end{tcolorbox}

\newpage

\subsection{Is SU(2) exactly equivalent to O(3)?}\label{sec:equivalence}
If we take the 2nd approach for Ambiguity-1 for SU(2) coherent state, bi-quadrutic and bi-cubic terms are transformed like eqs.\eqref{eq:SU2-quad2} and \eqref{eq:SU2-cubic2}.
In that approach, we completely ignore the algebraic relationship unique in SU(2), such as \eqref{eq:pauli}. Is this SU(2) approach exactly equivalent to O(3) vector formalism?
The answer is \emph{partially} yes. We will see this "partial" equivalence in this section.

SU(2) coherent state is explicitly written as
\begin{equation}
    \ket{\psi} = \cos \psi_1\ket{0} + e^{i\phi_1}\sin \psi_1\ket{1}.
\end{equation}
The expectation values of spin operators are 
\begin{align}
    \expval{S^x} &= \begin{pmatrix}\cos \psi_1 & e^{-i\phi_1}\sin \psi_1\end{pmatrix}\frac{1}{2}\begin{pmatrix}0 & 1 \\ 1 & 0 \end{pmatrix} \begin{pmatrix}\cos \psi_1 \\ e^{i\phi_1}\sin \psi_1\end{pmatrix} \\
    &= \frac{1}{2}\begin{pmatrix}\cos \psi_1 & e^{-i\phi_1}\sin \psi_1\end{pmatrix} \begin{pmatrix}e^{i\phi_1}\sin \psi_1 \\ \cos \psi_1 \end{pmatrix} \\
    &= \frac{1}{2}\left( e^{i\phi_1}\sin \psi_1 \cos \psi_1 + e^{-i\phi_1}\sin \psi_1 \cos \psi_1 \right) \\
    &= \frac{1}{2}\cos \phi_1 \sin 2\psi_1\label{eq:SU2-Sx}
\end{align}

\begin{align}
    \expval{S^y} &= \begin{pmatrix}\cos \psi_1 & e^{-i\phi_1}\sin \psi_1\end{pmatrix}\frac{1}{2}\begin{pmatrix}0 & -i \\ i & 0 \end{pmatrix} \begin{pmatrix}\cos \psi_1 \\ e^{i\phi_1}\sin \psi_1\end{pmatrix} \\
    &= \frac{i}{2}\begin{pmatrix}\cos \psi_1 & e^{-i\phi_1}\sin \psi_1\end{pmatrix} \begin{pmatrix}-e^{i\phi_1}\sin \psi_1 \\ \cos \psi_1 \end{pmatrix} \\
    &= \frac{i}{2}\left(-e^{i\phi_1}\sin \psi_1 \cos \psi_1 + e^{-i\phi_1}\sin \psi_1 \cos \psi_1 \right) \\
    &= \frac{1}{2}\sin \phi_1 \sin 2\psi_1\label{eq:SU2-Sy}
\end{align}

\begin{align}
    \expval{S^z} &= \begin{pmatrix}\cos \psi_1 & e^{-i\phi_1}\sin \psi_1\end{pmatrix}\frac{1}{2}\begin{pmatrix}1 & 0 \\ 0 & -1 \end{pmatrix} \begin{pmatrix}\cos \psi_1 \\ e^{i\phi_1}\sin \psi_1\end{pmatrix} \\
    &= \frac{1}{2} \left(\cos^2 \phi_1 - \sin^2 \psi_1 \right) \\
    &= \frac{1}{2} \cos 2\psi_1 \label{eq:SU2-Sz}
\end{align}
If we replace bi-quadrutic and bi-cubic term as square and cubic of dipole expectation values as eqs.\eqref{eq:SU2-quad2} and \eqref{eq:SU2-cubic2},
The model (formulation) is exactly equivalent to O(3) vector. Both have two variables at each site. However, as we can see in eqs. \eqref{eq:SU2-Sx}, \eqref{eq:SU2-Sy}, and \eqref{eq:SU2-Sz}, \textbf{\emph{dipole length of SU(2) is 1/2.}}
Usually we simulate the O(3) spins by setting each spin length 1, so this makes "partial" equivalence between O(3) and SU(2) formalism.

Using eqs. \eqref{eq:SU2-quad2} and \eqref{eq:SU2-cubic2}, our Hamiltonian is effectively written as
\begin{equation}
    H_{SU(2)} = J_{1}\expval{\hat{\bold{S}}_{i}\cdot \hat{\bold{S}}_{j}} + J_2 \expval{\hat{\bold{S}}_{i}\cdot \hat{\bold{S}}_{j}}^2 + J_3 \expval{\hat{\bold{S}}_{i}\cdot \hat{\bold{S}}_{j}}^3.
\end{equation}
When the system is completely ferromagnetic, for example, angles $\psi_1, \phi_1$ are independent of sites. The energy (per site) becomes
\begin{equation}
    E_{FM,SU(2)} = J_1 \times \frac{1}{4} + J_2 \times \left(\frac{1}{4}\right)^2 + J_3 \times \left(\frac{1}{4}\right)^3.
\end{equation}
While in O(3) vector with spin length 1, FM energy is
\begin{equation}
    E_{FM,O(3)} = J_1 \times 1 + J_2 \times 1^2 + J_3 \times 1^3.
\end{equation}
Hence, in this SU(2) formalism, bi-quadrutic and bi-cubic term is underestimated compared with O(3) vector. 
In contrast, if we regard SU(2) formalism as the more appripriate classical limit of the original Hamiltonian, O(3) simulation has to be done with each spin length 1/2.
Which is the better? It is useful to compare SU(4) results with O(3) results. Before that, let us consider some candidates.
The fiest candidate is just the length 1, which is widely used in classical spin research. Another one is length $1/2$ originating from SU(2).
The third one is $\sqrt{S(S+1)}$, because spin operators satisfy the identifierstyle
\begin{equation}
    \hat{S}_x^2 + \hat{S}_y^2 + \hat{S}_z^2 = S(S+1).
\end{equation}

\begin{figure}[t]
    \centering
    \includegraphics[width=0.6\columnwidth]{/Users/sogenikegami/Documents/Research/Kitaev-AKLT/SU_N_Coherent_jax/SU4/figure/Haar-random/6-96/Emin_d2E_2.pdf}
    \caption{Groud state energy curve for Kitaev-AKLT model by SU(4) coherent state.}
    \label{fig:SU4-K-AKLT}
\end{figure}

\begin{figure}[t]
    \centering
    \subfigure[{Length $1$}]{\includegraphics[width=0.45\textwidth]{/Users/sogenikegami/Documents/Research/Kitaev-AKLT/classical_jax/main/figure/length-1/neighbor/20-80-4/Emin_d2E.pdf}}
    \subfigure[{Length $1/2$}]{\includegraphics[width=0.45\textwidth]{/Users/sogenikegami/Documents/Research/Kitaev-AKLT/classical_jax/main/figure/length-half/half-length/Emin_d2E.pdf}} \\
    \subfigure[{Length $\sqrt{S(S+1)}=\sqrt{15}/4$}]{\includegraphics[width=0.45\textwidth]{/Users/sogenikegami/Documents/Research/Kitaev-AKLT/classical_jax/main/figure/sqrt15-2/6-96/Emin_d2E.pdf}}
    \subfigure[{Length $S=3/2$}]{\includegraphics[width=0.45\textwidth]{/Users/sogenikegami/Documents/Research/Kitaev-AKLT/classical_jax/main/figure/length-3_2/2-128/Emin_d2E.pdf}}
    \caption{Ground state energy curve for Kitaev-AKLT model by O(3) vector with various length.}
    \label{fig:O3-K-AKLT}
\end{figure}


In fact, Batista's paper \cite{PhysRevB.96.134408} employs this length to compare the results of their clasical model against the spin-$S$ quantum version of the model.
The last candidate is $S$. Length $\sqrt{S(S+1)}$ is based on the "expectation value of $\hat{\mathbf{S}}^2$", but in the coherent state simulation, dipole length is different from that, since 
the operation of taking expectation values and operator product is non-commutative. The certain expression is
\begin{equation}
    S(S+1) = \expval{\hat{S}_x^2 + \hat{S}_y^2 + \hat{S}_z^2} \neq \expval{\hat{S}_x}^2 + \expval{\hat{S}_y}^2 + \expval{\hat{S}_z}^2 \leq S.
\end{equation}
The O(3) vector model discards the multipole components and just approximate the dipole components of the originai model. In that sense, If the quantum or semi-classical ground states have
full-length dipole, length $S$ might be the most appropriate. 


Now it is the time to compare some numerical results. 
Fig.\ref{fig:SU4-K-AKLT} shows the ground state energy curve for Kitaev-AKLT model with SU(4) coherent state. 
The curve was calculated by JAX+Optax package in python, which enable us to utilize GPU and optimize the cost function at high speed.
Among the several phases in the phase diagram, the phases $\xi=0.5$ belongs to is just the dipole ferro magnetic phase with full length dipole $3/2$.
Comparing the energy around this parameter is the nice test which length is appropriate as the classical limit. 



Figs.\ref{fig:O3-K-AKLT} are the ground state energy curve for the same model by O(3) vector with four lengths.
Among four, length $S=3/2$ reproduces the same energy as the SU(4) around $\xi \sim 0.5$, 
indicating that around this point SU(4) result is just the pure dipole FM phase with length $S=3/2$.
The dipole length of this model was happened to be full-length $3/2$ for all parameters. In that sense, 
O(3) with length $3/2$ may be the best classical limit (dipole model) for this model.
We successfully catch the FM phase in dipole model (classical O(3) vector model). However, we could not for other phases.
Indeed, we confirmed that an intriguing phase appeared around $0 \lesssim \xi \lesssim 0.25$ for length $3/2$ and $\sqrt{S(S+1)}$. 
This phase was so-called non-coplanar phase, which is quite different from the usual Neel phase realized in SU(4) case.
We see energy discrepancy even though all the SU(4) ground state phases had full-length dipole.
This suggests that quadrupole or octupole conponent contribute to the energy and stabilizing Neel and other phases.
We automatically define the classical limit dipole model as the same Hamiltonian as the SU(4) model, 
but we may need to analytically calculate the dipole part Hamiltonian of the original model.

Bi-quadrutic and bi-cubic terms pose this problem. If we have only bi-linear term (Heisenberg, Kitaev, Gamma, DM...), energy scale is modified but other results are independent from this problem.

\newpage

\section{SU(2) Magnon}\label{sec:SU2_magnon}
In this section, we derive the SU(2) magnon dispersion for FM, Neel, zigzag states in Kitaev-AKLT model. 
Generally, if the Hamiltonian include biquadrutic or higher terms, this single SU(2) magnon cannot correctly derive the instability points \cite{10.1093/ptep/ptu109}.
However, we start from this simpler calculation.

\subsection{FM phase}
Holstein-Primakoff (HP) transformation is defined as
\begin{align}
    &S^z = S-\hat{a}^{\dagger}\hat{a}, \\
    &S^+ = S^x+iS^y = \sqrt{2S} \sqrt{1-\frac{\hat{n}}{2S}} \hat{a}, \\ 
    &S^- = S^x-iS^y = \sqrt{2S} \hat{a}^{\dagger} \sqrt{1-\frac{\hat{n}}{2S}},
\end{align}
where $\hat{a}^{\dagger}, \hat{a}$ operators are magnon creation/annihilation operators, $\hat{n}=\hat{a}^{\dagger}\hat{a}$ is an usual number operator of the boson.
$\hat{a}^\dagger, \hat{a},$ and $\hat{n}$ follows usual boson and harmonic oscilattor commutation relationship.
\begin{align} 
    \left[ \hat{a}, \hat{a}^\dagger \right] = 1, \quad \left[ \hat{n}, \hat{a}^\dagger \right] = \hat{a}^\dagger, \quad \left[ \hat{n}, \hat{a} \right] = -\hat{a}.
\end{align}
The honeycomb lattice has two sublattices within its unit cell, so we introduce two boson operators $\hat{a}$ and $\hat{b}$.
These different kinds of operators always commute with each other.
Then, the Heisenberg term between A and B sublattices is rewritten as
\begin{align}
    \hat{\bm{S}}_A \cdot \hat{\bm{S}}_B &= \hat{S}_A^z \hat{S}_B^z + \frac{1}{2} (\hat{S}^+_A \hat{S}^-_B + \hat{S}^-_A \hat{S}^+_B)  \notag \\
    &= (S-\hat{a}^\dagger_i \hat{a}_i)(S-\hat{b}^\dagger_i \hat{b}_i) + \frac{2S}{2} \left( \sqrt{1-\frac{\hat{a}^\dagger_i \hat{a}_i}{2S}} \hat{a}_i \hat{b}^{\dagger}_i \sqrt{1-\frac{\hat{b}^\dagger_i \hat{b}_i}{2S}} + \hat{a}^{\dagger}_i \sqrt{1-\frac{\hat{a}^\dagger_i \hat{a}_i}{2S}}   \sqrt{1-\frac{\hat{b}^\dagger_i \hat{b}_i}{2S}} \hat{b}_i  \right) \notag \\
    &\sim S^2 - S\left( \hat{a}^\dagger_i \hat{a}_i + \hat{b}^\dagger_i \hat{b}_i - \hat{a}_i \hat{b}^{\dagger}_i - \hat{a}^\dagger_i \hat{b}_i \right).
\end{align}
The subscript $i$ indicates the unit cell index. If we consider the Heisenberg interaction between A sub in $i$-th unit cell and B sub in $j$-th unit cell, we only replace the subsctipt of $\hat{b}$ operators from $i$ to $j$.
From the second line to the third line we ignore the magnon 2-body or higher interaction (e.g., $a^\dagger a^\dagger a b$). These higher interaction could be important for 
multipole fluctuation. In the generalized spin wave formalism we can deal with these effects as quadrutic terms, instead of introducing many kinds of bosons.

From above calculation, we get linear terms for both biquadrutic and bicubic interactions as follows.
\begin{align}
    \left( \hat{\bm{S}}_A \cdot \hat{\bm{S}}_B \right)^2  &\sim \left(S^2 - S\left( \hat{a}^\dagger_i \hat{a}_i + \hat{b}^\dagger_i \hat{b}_i - \hat{a}_i \hat{b}^{\dagger}_i - \hat{a}^\dagger_i \hat{b}_i \right) \right)\left( S^2 - S\left( \hat{a}^\dagger_i \hat{a}_i + \hat{b}^\dagger_i \hat{b}_i - \hat{a}_i \hat{b}^{\dagger}_i - \hat{a}^\dagger_i \hat{b}_i \right) \right) \notag  \\
    &\sim S^4 - 2S^3 \left( \hat{a}^\dagger_i \hat{a}_i + \hat{b}^\dagger_i \hat{b}_i - \hat{a}_i \hat{b}^{\dagger}_i - \hat{a}^\dagger_i \hat{b}_i \right),
\end{align}

\begin{align}
    \left( \hat{\bm{S}}_A \cdot \hat{\bm{S}}_B \right)^3  &\sim \left(S^2 - S\left( \hat{a}^\dagger_i \hat{a}_i + \hat{b}^\dagger_i \hat{b}_i - \hat{a}_i \hat{b}^{\dagger}_i - \hat{a}^\dagger_i \hat{b}_i \right) \right)\left(S^4 - 2S^3 \left( \hat{a}^\dagger_i \hat{a}_i + \hat{b}^\dagger_i \hat{b}_i - \hat{a}_i \hat{b}^{\dagger}_i - \hat{a}^\dagger_i \hat{b}_i \right) \right) \notag \\
    &\sim S^6 - 3S^5 \left( \hat{a}^\dagger_i \hat{a}_i + \hat{b}^\dagger_i \hat{b}_i - \hat{a}_i \hat{b}^{\dagger}_i - \hat{a}^\dagger_i \hat{b}_i \right).
\end{align}
In the usual spin wave theory, we assume that $S$ is enough large that we can ignore $1/S$ contribution which appears in 2-body or many-body magnon interactions.
In that sense, it is unclear whether the $S^5$ contribution is reasonable or not.
Denoting $\hat{h}_{\rm Heis}^{i,j} = \left( \hat{a}^\dagger_i \hat{a}_i + \hat{b}^\dagger_j \hat{b}_j - \hat{a}_i \hat{b}^{\dagger}_j - \hat{a}^\dagger_i \hat{b}_j \right)$, the Heisenberg interaction around the A sub site in the unit cell $i$
is 
\begin{align}
    \hat{h}_{\rm Heis}^i &= \left( \hat{\bm{S}}_{i,A} \cdot \hat{\bm{S}}_{i,B} \right) + \left( \hat{\bm{S}}_{i,A} \cdot \hat{\bm{S}}_{i-1,B} \right) + \left( \hat{\bm{S}}_{i,A} \cdot \hat{\bm{S}}_{i-2,B} \right) \notag \\
    &\sim 3S^2 - S \left( \hat{h}_{\rm Heis}^{i,i} + \hat{h}_{\rm Heis}^{i,i-1} + \hat{h}_{\rm Heis}^{i,i-2} \right).
\end{align}
The AKLT interaction at A sub site in the unit cell $i$ is
\begin{equation}
    \hat{h}_{\rm AKLT}^{i} \sim 3S^2\left( 1+\frac{116}{243}S^2 + \frac{16}{243}S^4 \right) - S\left( 1+\frac{116}{243}2S^2 + \frac{16}{243}3S^4 \right)\left( \hat{h}_{\rm Heis}^{i,i} + \hat{h}_{\rm Heis}^{i,i-1} + \hat{h}_{\rm Heis}^{i,i-2} \right).
\end{equation}

The Kitaev interaction is slightly trickey. 
We assume that A sub in the unit cell $i (\bm{R}_i)$  has $S^zS^z$ interaction with B sub in the same unit cell, while $S^xS^x$ with B sub in the unit cell $i-1: \bm{R}_i-\bm{a}_1$ and $S^yS^y$ with B sub in the unit cell $i-2: \bm{R}_i-\bm{a}_2$.
Then, the Kitaev interaction is rewritten as
\begin{align}
    \hat{H}_{\rm Kitaev}^i &= \hat{S}^z_{i,A}\hat{S}^z_{i,B} + \hat{S}^x_{i,A}\hat{S}^x_{i-1,B} + \hat{S}^y_{i,A}\hat{S}^y_{i-2,B} \notag  \\
    &=  \hat{S}^z_{i,A}\hat{S}^z_{i,B} + \frac{\hat{S}^+_{i,A} + \hat{S}^-_{i,A}}{2}\frac{\hat{S}^+_{i-1,B} + \hat{S}^-_{i-1,B}}{2} + \frac{\hat{S}^+_{i,A} - \hat{S}^-_{i,A}}{2i}\frac{\hat{S}^+_{i-2, B} - \hat{S}^-_{i-2, B}}{2i} \notag\\
    &= (S-\hat{a}^\dagger_i \hat{a}_i)(S-\hat{b}^\dagger_i \hat{b}_i)  \notag \\ 
    & \quad  +\frac{2S}{4} \left(  \sqrt{1-\frac{\hat{a}^\dagger_i \hat{a}_i}{2S}} \hat{a}_i + \hat{a}^{\dagger}_i \sqrt{1-\frac{\hat{a}^\dagger_i \hat{a}_i}{2S}} \right) \left(  \sqrt{1-\frac{\hat{b}^\dagger_{i-1} \hat{b}_{i-1}}{2S}} \hat{b}_{i-1} + \hat{b}^{\dagger}_{i-1} \sqrt{1-\frac{\hat{b}^\dagger_{i-1} \hat{b}_{i-1}}{2S}} \right) \notag \\ 
    & \quad  -\frac{2S}{4} \left(  \sqrt{1-\frac{\hat{a}^\dagger_i \hat{a}_i}{2S}} \hat{a}_i - \hat{a}^{\dagger}_i \sqrt{1-\frac{\hat{a}^\dagger_i \hat{a}_i}{2S}} \right) \left(  \sqrt{1-\frac{\hat{b}^\dagger_{i-2} \hat{b}_{i-2}}{2S}} \hat{b}_{i-2} - \hat{b}^{\dagger}_{i-2} \sqrt{1-\frac{\hat{b}^\dagger_{i-2} \hat{b}_{i-2}}{2S}} \right) \notag \\
    &\sim \left( S^2 -S \hat{a}^\dagger_i \hat{a}_i -S\hat{b}^\dagger_i \hat{b}_i   \right)  \notag \\ 
    & \quad \quad  + \frac{S}{2} \left( \hat{a}_i\hat{b}_{i-1} + \hat{a}_i\hat{b}_{i-1}^\dagger + \hat{a}_i^\dagger \hat{b}_{i-1} + \hat{a}_i^\dagger\hat{b}_{i-1}^\dagger \right) \notag \\
    & \quad \quad  - \frac{S}{2} \left( \hat{a}_i\hat{b}_{i-2} - \hat{a}_i\hat{b}_{i-2}^\dagger - \hat{a}_i^\dagger \hat{b}_{i-2} + \hat{a}_i^\dagger\hat{b}_{i-2}^\dagger \right). 
\end{align} 

In total, our Hamiltonian is reformulated using the HP bosons in the linear range as
\begin{align}
    H &= \cos(2\pi \xi)H_{\rm AKLT} + \sin(2\pi \xi)H_{\rm Kitaev} \notag \\
    &=\cos(2\pi \xi)\sum_{i}h_{\rm AKLT}^{i} + \sin(2\pi \xi)\sum_{i}\hat{H}_{\rm Kitaev}^i \notag \\
    &\sim C_{\rm const}(\xi) + C_{\rm AKLT}(\xi)\sum_{i}\left( \hat{h}_{\rm Heis}^{i,i} + \hat{h}_{\rm Heis}^{i,i-1} + \hat{h}_{\rm Heis}^{i,i-2} \right) + C_{\rm Kitaev} \sum_{i} \hat{h}_{\rm Kitaev}^i.
\end{align}
We summarize the Hamiltonian and the definition of the each term in a box below.
\begin{tcolorbox}
    \textbf{FM SU(2) magnon Hamiltonian}

    Under FM assumption the Kitaev-AKLT Hamiltonian is rewritten by HP bosons as follows.
    \begin{align}
        H &= \cos(2\pi \xi)H_{\rm AKLT} + \sin(2\pi \xi)H_{\rm Kitaev} \notag \\
        &\sim C_{\rm const}(\xi) + C_{\rm AKLT}(\xi)\sum_{i}\left( \hat{h}_{\rm Heis}^{i,i} + \hat{h}_{\rm Heis}^{i,i-1} + \hat{h}_{\rm Heis}^{i,i-2} \right) + C_{\rm Kitaev} \sum_{i} \hat{h}_{\rm Kitaev}^i, \label{eq:Hamiltonian_LSWT_FM}
    \end{align}
    with
    \begin{align}
        &\hat{h}_{\rm Heis}^{i,j} = \hat{a}^\dagger_i \hat{a}_i + \hat{b}^\dagger_j \hat{b}_j - \hat{a}_i \hat{b}^{\dagger}_j - \hat{a}^\dagger_i \hat{b}_j, \label{eq:hHeis} \\
        &\hat{h}_{\rm Kitaev}^i =  \hat{a}^\dagger_i \hat{a}_i + \hat{b}^\dagger_j \hat{b}_j -\frac{1}{2}\left( \hat{a}_i\hat{b}_{i-1} + \hat{a}_i\hat{b}_{i-1}^\dagger + \hat{a}_i^\dagger \hat{b}_{i-1} + \hat{a}_i^\dagger\hat{b}_{i-1}^\dagger \right) \label{eq:hKitaev} \notag \\ 
        & \quad \quad \quad \quad \quad \quad \quad \quad \quad + \frac{1}{2} \left( \hat{a}_i\hat{b}_{i-2} - \hat{a}_i\hat{b}_{i-2}^\dagger - \hat{a}_i^\dagger \hat{b}_{i-2} + \hat{a}_i^\dagger\hat{b}_{i-2}^\dagger \right),
    \end{align}
    and
    \begin{align}
        &C_{\rm const}(\xi) = \sum_{i}3S^2\left( 1+\frac{116}{243}S^2 + \frac{16}{243}S^4 \right) \cos(2\pi \xi) + \sum_{i} S^2\sin(2\pi \xi), \\
        &C_{\rm AKLT}(\xi) = - S\left( 1+\frac{116}{243}2S^2 + \frac{16}{243}3S^4 \right)\cos(2\pi \xi), \\
        &C_{\rm Kitaev}(\xi) = -S \sin(2\pi \xi).
    \end{align}
    The summation runs for all the unit cells $i$.
\end{tcolorbox}

The Hamiltonian can be diagonalized by the usual procedure: the Fourier transformation and the Bogoliubov transformation.
We define the Fourier transformation of the HP bosons as
\begin{align} 
    &\hat{a}_{\bm{k}} = \frac{1}{\sqrt{N}}\sum_{i}e^{-i\bm{k}\cdot\bm{R}_i}\hat{a}_i, \quad  \hat{a}^\dagger_{\bm{k}} = \frac{1}{\sqrt{N}}\sum_{i}e^{+i\bm{k}\cdot\bm{R}_i}\hat{a}^\dagger_i, \label{eq:FT-HPboson-a} \\
    &\hat{b}_{\bm{k}} = \frac{1}{\sqrt{N}}\sum_{i}e^{-i\bm{k}\cdot\bm{R}_i}\hat{b}_i, \quad  \hat{b}^\dagger_{\bm{k}} = \frac{1}{\sqrt{N}}\sum_{i}e^{+i\bm{k}\cdot\bm{R}_i}\hat{b}^\dagger_i, \label{eq:FT-HPboson-b}
\end{align}
and the inverse transformation as
\begin{align}
    &\hat{a}_i = \frac{1}{\sqrt{N}}\sum_{i}e^{+i\bm{k}\cdot\bm{R}_i}\hat{a}_{\bm{k}}, \quad  \hat{a}^\dagger_i = \frac{1}{\sqrt{N}}\sum_{i}e^{-i\bm{k}\cdot\bm{R}_i}\hat{a}^\dagger_{\bm{k}}, \\
    &\hat{b}_i = \frac{1}{\sqrt{N}}\sum_{i}e^{+i\bm{k}\cdot\bm{R}_i}\hat{b}_{\bm{k}}, \quad  \hat{b}^\dagger_i = \frac{1}{\sqrt{N}}\sum_{i}e^{-i\bm{k}\cdot\bm{R}_i}\hat{b}^\dagger_{\bm{k}},
\end{align}
where the summation runs for all the unit cells and $N$ denotes the number of the unit cell.
The boson interaction terms are transformed as follows.
\begin{align}
    \sum_{i}\hat{a}^\dagger_i \hat{a}_i = \sum_{\bm{k}}\hat{a}^\dagger_{\bm{k}} \hat{a}_{\bm{k}} = \sum_{\bm{k}}\hat{a}^\dagger_{\bm{-k}} \hat{a}_{\bm{-k}}, & \quad \sum_{i}\hat{b}^\dagger_i \hat{b}_i = \sum_{\bm{k}}\hat{b}^\dagger_{\bm{k}} \hat{b}_{\bm{k}}, \\
    \sum_{i}\hat{a}_i \hat{b}_i^\dagger = \sum_{\bm{k}}\hat{a}_{\bm{k}} \hat{b}_{\bm{k}}^\dagger =\sum_{\bm{k}}\hat{a}_{-\bm{k}} \hat{b}_{-\bm{k}}^\dagger, & \quad \sum_{i}\hat{a}^\dagger_i \hat{b}_i = \sum_{\bm{k}}\hat{a}^\dagger_{\bm{k}} \hat{b}_{\bm{k}}, \\
    \sum_{i}\hat{a}_i \hat{b}_{i-1}^\dagger = \sum_{\bm{k}} e^{+i\bm{k}\cdot \bm{a}_1}\hat{a}_{\bm{k}} \hat{b}_{\bm{k}}^\dagger = \sum_{\bm{k}} e^{\red{-}i\bm{k}\cdot \bm{a}_1}\hat{a}_{-\bm{k}} \hat{b}_{-\bm{k}}^\dagger , & \quad \sum_{i}\hat{a}^\dagger_{i} \hat{b}_{i-1} = \sum_{\bm{k}}e^{-i\bm{k}\cdot \bm{a}_1} \hat{a}^\dagger_{\bm{k}} \hat{b}_{\bm{k}}, \\
    \sum_{i}\hat{a}_i \hat{b}_{i-2}^\dagger = \sum_{\bm{k}} e^{+i\bm{k}\cdot \bm{a}_2}\hat{a}_{\bm{k}} \hat{b}_{\bm{k}}^\dagger = \sum_{\bm{k}} e^{\red{-}i\bm{k}\cdot \bm{a}_2}\hat{a}_{-\bm{k}} \hat{b}_{-\bm{k}}^\dagger, & \quad \sum_{i}\hat{a}^\dagger_{i} \hat{b}_{i-2} = \sum_{\bm{k}}e^{-i\bm{k}\cdot \bm{a}_2} \hat{a}^\dagger_{\bm{k}} \hat{b}_{\bm{k}}, 
\end{align}
and 
\begin{align}
    \sum_{i}\hat{a}_i \hat{b}_{i-1} = \sum_{\bm{k}}e^{+i\bm{k}\cdot \bm{a}_1} \hat{a}_{\bm{k}} \hat{b}_{-\bm{k}} = \sum_{\bm{k}}e^{\red{-}i\bm{k}\cdot \bm{a}_1} \hat{a}_{-\bm{k}} \hat{b}_{\bm{k}}, & \quad \sum_{i}\hat{a}_i^\dagger \hat{b}_{i-1}^\dagger = \sum_{\bm{k}}e^{-i\bm{k}\cdot \bm{a}_1} \hat{a}_{\bm{k}}^\dagger \hat{b}_{-\bm{k}}^\dagger, \\
    \sum_{i}\hat{a}_i \hat{b}_{i-2} = \sum_{\bm{k}}e^{+i\bm{k}\cdot \bm{a}_2} \hat{a}_{\bm{k}} \hat{b}_{-\bm{k}} = \sum_{\bm{k}}e^{\red{-}i\bm{k}\cdot \bm{a}_2} \hat{a}_{-\bm{k}} \hat{b}_{\bm{k}}, & \quad \sum_{i}\hat{a}_i^\dagger \hat{b}_{i-2}^\dagger = \sum_{\bm{k}}e^{-i\bm{k}\cdot \bm{a}_2} \hat{a}_{\bm{k}}^\dagger \hat{b}_{-\bm{k}}^\dagger.
\end{align}

The AKLT interaction in eq.\eqref{eq:Hamiltonian_LSWT_FM} is transformed as
\begin{align}
    &\sum_{i}\left( \hat{h}_{\rm Heis}^{i,i} + \hat{h}_{\rm Heis}^{i,i-1} + \hat{h}_{\rm Heis}^{i,i-2} \right) \notag  \\
    & \quad = \sum_{i} \left( 3\hat{a}^\dagger_i \hat{a}_i + 3\hat{b}^\dagger_i \hat{b}_i - \left( \hat{a}_i \hat{b}^{\dagger}_i + \hat{a}_i \hat{b}^{\dagger}_{i-1} + \hat{a}_i \hat{b}^{\dagger}_{i-2}\right) - \left( \hat{a}^\dagger_i \hat{b}_i + \hat{a}^\dagger_i \hat{b}_{i-1} + \hat{a}^\dagger_i \hat{b}_{i-2} \right) \right) \notag \\
    & \quad = \sum_{\bm{k}} \left( 3\hat{a}_{\bm{k}}^\dagger\hat{a}_{\bm{k}} + 3\hat{b}_{\bm{k}}^\dagger\hat{b}_{\bm{k}} - (1+e^{+i\bm{k}\cdot \bm{a}_1}+e^{+i\bm{k}\cdot \bm{a}_2}) \hat{a}_{\bm{k}} \hat{b}_{\bm{k}}^\dagger  - (1+e^{-i\bm{k}\cdot \bm{a}_1}+e^{-i\bm{k}\cdot \bm{a}_2}) \hat{a}_{\bm{k}}^\dagger  \hat{b}_{\bm{k}}\right) \notag \\
    & \quad = \sum_{\bm{k}} \left( 3\hat{a}_{\bm{k}}^\dagger\hat{a}_{\bm{k}} + 3\hat{b}_{\bm{k}}^\dagger\hat{b}_{\bm{k}} - \Gamma^*(\bm{k}) \hat{a}_{\bm{k}} \hat{b}_{\bm{k}}^\dagger  - \Gamma(\bm{k}) \hat{a}_{\bm{k}}^\dagger  \hat{b}_{\bm{k}}\right) \notag \\
    & \quad = \sum_{\bm{k}} \left( \frac{3}{2}\hat{a}_{\bm{k}}^\dagger\hat{a}_{\bm{k}} + \frac{3}{2}\hat{a}_{-\bm{k}}^\dagger\hat{a}_{-\bm{k}} + \frac{3}{2}\hat{b}_{\bm{k}}^\dagger\hat{b}_{\bm{k}} + \frac{3}{2}\hat{b}_{-\bm{k}}^\dagger\hat{b}_{-\bm{k}} \right.  \notag \\
    & \quad \quad \quad \quad \quad \quad \left. - \frac{\Gamma(\bm{k})}{2} \hat{a}_{-\bm{k}} \hat{b}_{-\bm{k}}^\dagger - \frac{\Gamma^*(\bm{k})}{2} \hat{a}_{\bm{k}} \hat{b}_{\bm{k}}^\dagger  - \frac{\Gamma(\bm{k})}{2} \hat{a}_{\bm{k}}^\dagger  \hat{b}_{\bm{k}} - \frac{\Gamma^*(\bm{k})}{2} \hat{a}_{-\bm{k}}^\dagger  \hat{b}_{-\bm{k}}\right).
\end{align}
In the last line we defined $\Gamma(\bm{k}) = 1+e^{-i\bm{k}\cdot \bm{a}_1}+e^{-i\bm{k}\cdot \bm{a}_2}$.

The Kitaev interaction in eq.\eqref{eq:Hamiltonian_LSWT_FM} is transformed as
\begin{align}
    \sum_{i}\hat{h}_{\rm Kitaev}^i &= \sum_{i} \hat{a}^\dagger_i \hat{a}_i + \hat{b}^\dagger_j \hat{b}_j -\frac{1}{2}\left( \hat{a}_i\hat{b}_{i-1} + \hat{a}_i\hat{b}_{i-1}^\dagger + \hat{a}_i^\dagger \hat{b}_{i-1} + \hat{a}_i^\dagger\hat{b}_{i-1}^\dagger \right) \notag \\ 
        & \quad \quad \quad \quad \quad \quad \quad  + \frac{1}{2} \left( \hat{a}_i\hat{b}_{i-2} - \hat{a}_i\hat{b}_{i-2}^\dagger - \hat{a}_i^\dagger \hat{b}_{i-2} + \hat{a}_i^\dagger\hat{b}_{i-2}^\dagger \right) \notag  \\
        &= \sum_{\bm{k}} \hat{a}_{\bm{k}}^\dagger\hat{a}_{\bm{k}} + \hat{b}_{\bm{k}}^\dagger\hat{b}_{\bm{k}} \notag \\ 
        & \quad \quad -\frac{1}{2}\left( e^{+i\bm{k}\cdot\bm{a}_1} - e^{+i\bm{k}\cdot\bm{a}_2} \right) \hat{a}_{\bm{k}}\hat{b}_{-\bm{k}} -\frac{1}{2}\left( e^{+i\bm{k}\cdot\bm{a}_1} + e^{+i\bm{k}\cdot\bm{a}_2} \right) \hat{a}_{\bm{k}}\hat{b}_{\bm{k}}^\dagger \notag \\ 
        & \quad \quad -\frac{1}{2}\left( e^{-i\bm{k}\cdot\bm{a}_1} + e^{-i\bm{k}\cdot\bm{a}_2} \right) \hat{a}_{\bm{k}}^\dagger \hat{b}_{\bm{k}} -\frac{1}{2}\left( e^{-i\bm{k}\cdot\bm{a}_1} - e^{-i\bm{k}\cdot\bm{a}_2} \right) \hat{a}_{\bm{k}}^\dagger \hat{b}_{-\bm{k}}^\dagger \notag \\ 
        &= \sum_{\bm{k}}\left( \hat{a}_{\bm{k}}^\dagger\hat{a}_{\bm{k}} + \hat{b}_{\bm{k}}^\dagger\hat{b}_{\bm{k}} - f^*(\bm{k})\hat{a}_{\bm{k}}\hat{b}_{-\bm{k}} - g^*(\bm{k})\hat{a}_{\bm{k}}\hat{b}_{\bm{k}}^\dagger - g(\bm{k})\hat{a}_{\bm{k}}^\dagger \hat{b}_{\bm{k}} - f(\bm{k})\hat{a}_{\bm{k}}^\dagger\hat{b}_{-\bm{k}}^\dagger \right) \notag \\
        &= \sum_{\bm{k}} \frac{1}{2}\left( \hat{a}_{\bm{k}}^\dagger\hat{a}_{\bm{k}} + \hat{a}_{-\bm{k}}^\dagger\hat{a}_{-\bm{k}} + \hat{b}_{\bm{k}}^\dagger\hat{b}_{\bm{k}} + \hat{b}_{-\bm{k}}^\dagger\hat{b}_{-\bm{k}}  \right. \notag \\
        & \quad \quad \quad \quad \quad \quad  - f^*(\bm{k})\hat{a}_{\bm{k}}\hat{b}_{-\bm{k}} - f(\bm{k})\hat{a}_{-\bm{k}}\hat{b}_{\bm{k}} - g^*(\bm{k})\hat{a}_{\bm{k}}\hat{b}_{\bm{k}}^\dagger - g(\bm{k})\hat{a}_{-\bm{k}}\hat{b}_{-\bm{k}}^\dagger \notag \\
        & \quad \quad \quad \quad \quad \quad \left.- g(\bm{k})\hat{a}_{\bm{k}}^\dagger \hat{b}_{\bm{k}} - g^*(\bm{k}) \hat{a}_{-\bm{k}}^\dagger \hat{b}_{-\bm{k}} - f(\bm{k})\hat{a}_{\bm{k}}^\dagger\hat{b}_{-\bm{k}}^\dagger - f^*(\bm{k})\hat{a}_{-\bm{k}}^\dagger\hat{b}_{\bm{k}}^\dagger \right).
\end{align}
In the last line we defined $f(\bm{k}) = \frac{1}{2}\left( e^{-i\bm{k}\cdot\bm{a}_1} - e^{-i\bm{k}\cdot\bm{a}_2} \right)$ and $g(\bm{k}) = \frac{1}{2}\left( e^{-i\bm{k}\cdot\bm{a}_1} + e^{-i\bm{k}\cdot\bm{a}_2} \right)$.
Finally, we get the FM SU(2) magnon Hamiltonian
\begin{align}
    H &= \cos(2\pi \xi)H_{\rm AKLT} + \sin(2\pi \xi)H_{\rm Kitaev} \notag \\
    &\sim C_{\rm const}(\xi) + C_{\rm AKLT}(\xi)\sum_{i}\left( \hat{h}_{\rm Heis}^{i,i} + \hat{h}_{\rm Heis}^{i,i-1} + \hat{h}_{\rm Heis}^{i,i-2} \right) + C_{\rm Kitaev} \sum_{i} \hat{h}_{\rm Kitaev}^i. \label{eq:FM_SU2_magnon_H_FT}
\end{align}

It is convenient to write it down in the bilinear form with BdG like matrix.
Since eq.\eqref{eq:FM_SU2_magnon_H_FT} contains $-\bm{k}$ components, we have to prepare four component basis like
\begin{equation}
    \left( \hat{a}^\dagger_{\bm{k}}, \hat{b}^\dagger_{\bm{k}}, \hat{a}_{-\bm{k}}, \hat{b}_{-\bm{k}}\right) \quad \text{and} \quad \left( \hat{a}_{\bm{k}}, \hat{b}_{\bm{k}}, \hat{a}_{-\bm{k}}^\dagger, \hat{b}_{-\bm{k}}^\dagger\right)^T.
\end{equation}

\begin{tcolorbox}
    \textbf{FM SU(2) magnon Hamiltonian (after Fourier transformation)}
    Using this basis, our Hamiltonian \eqref{eq:FM_SU2_magnon_H_FT} is summarized as follows.
    \begin{equation}\label{eq:BdG-form}
        H = \sum_{\bm{k}}\left( \hat{a}^\dagger_{\bm{k}}, \hat{b}^\dagger_{\bm{k}}, \hat{a}_{-\bm{k}}, \hat{b}_{-\bm{k}}\right) \mathcal{H}(\bm{k}) \begin{pmatrix}
            \hat{a}_{\bm{k}} \\ \hat{b}_{\bm{k}} \\ \hat{a}_{-\bm{k}}^\dagger \\ \hat{b}_{-\bm{k}}^\dagger
        \end{pmatrix}
    \end{equation}
    with 
    \begin{equation}
        \mathcal{H}(\bm{k}) = \frac{1}{2}\begin{pmatrix}
            3C_{\rm A} + C_{\rm K} & -C_{\rm A}\Gamma-C_{\rm K}g & 0 & -C_{\rm K}f \\
            -C_{\rm A}\Gamma^* - C_{\rm K}g^* & 3C_{\rm A} + C_{\rm K} & -C_{\rm K}f^* & 0 \\
            0 & -C_{\rm K}f & 3C_{\rm A} + C_{\rm K} & -C_{\rm A}\Gamma - C_{\rm K}g \\
            -C_{\rm K}f^* & 0 & -C_{\rm A}\Gamma^*-C_{\rm K}g^* & 3C_{\rm A} + C_{\rm K}
        \end{pmatrix}.
    \end{equation}
    The definitions of some functions are
    \begin{align}
        &\Gamma(\bm{k}) = 1+e^{-i\bm{k}\cdot \bm{a}_1}+e^{-i\bm{k}\cdot \bm{a}_2}, \\
        &f(\bm{k}) = \frac{1}{2}\left( e^{-i\bm{k}\cdot\bm{a}_1} - e^{-i\bm{k}\cdot\bm{a}_2} \right), \\
        &g(\bm{k}) = \frac{1}{2}\left( e^{-i\bm{k}\cdot\bm{a}_1} + e^{-i\bm{k}\cdot\bm{a}_2} \right),
    \end{align}
    and 
    \begin{equation}
        \bm{a}_1 = \left( \frac{\sqrt{3}}{2}a, \frac{3}{2}a \right), \quad \bm{a}_2 = \left( -\frac{\sqrt{3}}{2}a, \frac{3}{2}a \right),
    \end{equation}
    where $a$ is the bond length between A and B sublattices.
    Two $2\times 2$ matrices 
    \begin{equation}
        h \equiv \frac{1}{2}\begin{pmatrix} 3C_{\rm A} + C_{\rm K} & -C_{\rm A}\Gamma-C_{\rm K}g \\ -C_{\rm A}\Gamma^* - C_{\rm K}g^* & 3C_{\rm A} + C_{\rm K} \end{pmatrix} \ \text{and} \ 
        \Delta \equiv \frac{1}{2}\begin{pmatrix} 0 & -C_{\rm K}f \\ -C_{\rm K}f^* & 0 \end{pmatrix}
    \end{equation}
    make the $4\times4$ matrix simple:
    \begin{equation}
        \mathcal{H}(\bm{k}) = \begin{pmatrix} h & \Delta \\ \Delta & h \end{pmatrix}.
    \end{equation}
\end{tcolorbox}

Note that biquadrutic and bicubit terms basically yield the same bilinear boson terms as the bilinear one,
and they only modify the coefficients of $\hat{h}_{\rm Heis}$. This makes the single SU(2) magnon description worse.
As I mentioned before, we have to include the magnon-magnon interaction terms or consider generalized spin wave theory.
The same happens for Neel assumed calculation.

The last task is to diagonalize the $4\times 4$ matrix. This unitary operation is called Bogoliubov transformation.
When we rely on numerical diagonalization, we have to multiply $\sigma_z$ (a diagonal matrix which takes $+1$ for creation operators and $-1$ for annihilation operators)
to our Hamiltonian. It is related to the paraunitary of the Bogoliubov transformtion matrix.
This multiplication makes our Hamiltonian non-Hermitian and its diagonalization means solving the generalized eigenvalue problems.
This operation is briefly understood like this: first we consider the path integral with action
\begin{equation}
    Z = \int \mathcal{D}\phi \exp (iS), \quad  S=\int  \hat{a}^\dagger \partial_\tau \hat{a} +  \hat{b}^\dagger \partial_\tau \hat{b} - H, \quad H = \begin{pmatrix} \hat{b}^\dagger & \hat{a} \end{pmatrix} \mathcal{H} \begin{pmatrix}\hat{b} \\ \hat{a}^\dagger \end{pmatrix}.
\end{equation}
Performing the partial integration for the time derivatice terms we get (after Fourier transformation)
\begin{equation}
    S = \int \begin{pmatrix}\hat{b}^\dagger & \hat{a} \end{pmatrix} \left[ \begin{pmatrix}
        -\omega & 0 \\ 0 & \omega
    \end{pmatrix} - \mathcal{H} \right]  \begin{pmatrix}\hat{b} \\ \hat{a}^\dagger \end{pmatrix} = \int \begin{pmatrix}\hat{b}^\dagger & \hat{a} \end{pmatrix} \sigma_z \left[ \omega I - \sigma_{z} \mathcal{H} \right]  \begin{pmatrix}\hat{b} \\ \hat{a}^\dagger \end{pmatrix}.
\end{equation}
The last one is reduced to the generalized eigenvalue problems.
When you write code by Python or Julia or other language, you will use diagonalization function (package).
Please take care not to use diagonalization packages for Hermitian matrices.


\subsection{Neel phase}
We assume Neel ordered state and derive the SU(2) single magnon band.
Basically we can perform the same procedure to calculate the magnon band as the FM case.
However, we have to modify the definition of HP transformation and bosons for down spins.
Let A sub points $+z$ direction and B sub points $-z$ direction. We rotate the quantization axis of the B sublattice by $\pi$ around $x$ axis by
\begin{align}
    &\hat{S}^x \rightarrow \hat{S}^{x\prime} = \hat{S}^x, \\ 
    &\hat{S}^y \rightarrow \hat{S}^{y\prime} =\hat{U}_x(\pi) \hat{S}^y \hat{U}_x(\pi)^\dagger = e^{-i\pi \hat{S}^x}\hat{S}^ye^{+i\pi \hat{S}^x} = -\hat{S}^y,  \\
    &\hat{S}^z \rightarrow \hat{S}^{z\prime} =\hat{U}_x(\pi) \hat{S}^y \hat{U}_x(\pi)^\dagger = e^{-i\pi \hat{S}^x}\hat{S}^ze^{+i\pi \hat{S}^x} = -\hat{S}^z. 
\end{align}
$\hat{S}^+$ and $\hat{S}^-$ are also modified correspondingly. 
THe Heisenberg interaction term in the rotated frame is, for example, 
\begin{equation}
    (\hat{\bm{S}}_A\cdot \hat{\bm{S}}_B) \rightarrow (\hat{\bm{S}}_A^{\prime}\cdot \hat{\bm{S}}_B^{\prime}) = (\hat{S}_A^x\hat{S}_B^x - \hat{S}_A^y\hat{S}_B^y - \hat{S}_A^z\hat{S}_B^z).
\end{equation}
From this rotation we can define the HP boson at B sublattice as
\begin{align}
    &\hat{S}^{z\prime} = -\hat{S}^{z} = -S + \hat{b}^\dagger\hat{b} \\
    &\hat{S}^{+\prime} = \hat{S}^{x\prime} + i\hat{S}^{y\prime} = \hat{S}^{x} -i\hat{S}^{y} = \sqrt{2S} \hat{b}^{\dagger} \sqrt{1-\frac{\hat{n}}{2S}} \\
    &\hat{S}^{-\prime} = \hat{S}^{x\prime} - i\hat{S}^{y\prime} = \hat{S}^x + i\hat{S}^y = \sqrt{2S} \sqrt{1-\frac{\hat{n}}{2S}} \hat{b}, 
\end{align}
We do not explicitly put prime mark on the spin operators from now on. 
The bilinear, biquadrutic, bicubic interactions are, up to linear terms,
\begin{align}
    \hat{\bm{S}}_A \cdot \hat{\bm{S}}_B &= \hat{S}_A^z \hat{S}_B^z + \frac{1}{2} (\hat{S}^+_A \hat{S}^-_B + \hat{S}^-_A \hat{S}^+_B) \notag \\
    &= (S-\hat{a}^\dagger_i \hat{a}_i)(\hat{b}^\dagger_i \hat{b}_i-S) + \frac{2S}{2} \left( \sqrt{1-\frac{\hat{a}^\dagger_i \hat{a}_i}{2S}} \hat{a}_i \sqrt{1-\frac{\hat{b}^\dagger_i \hat{b}_i}{2S}} \hat{b}_i  + \hat{a}^{\dagger}_i \sqrt{1-\frac{\hat{a}^\dagger_i \hat{a}_i}{2S}}    \hat{b}^{\dagger}_i \sqrt{1-\frac{\hat{b}^\dagger_i \hat{b}_i}{2S}} \right) \notag \\
    &\sim -S^2 + S\left( \hat{a}^\dagger_i \hat{a}_i + \hat{b}^\dagger_i \hat{b}_i + \hat{a}_i \hat{b}_i + \hat{a}^\dagger_i \hat{b}_i^{\dagger} \right)  \notag\\
    &= -S^2 + S\hat{h}_{\rm Heis}^{i,i},
\end{align}
\begin{align}
    \left( \hat{\bm{S}}_A \cdot \hat{\bm{S}}_B \right)^2 \sim S^4 -2S^3 \hat{h}_{\rm Heis}^{i,i},
\end{align}
\begin{align}
    \left( \hat{\bm{S}}_A \cdot \hat{\bm{S}}_B \right)^3 \sim -S^6 + 3S^5 \hat{h}_{\rm Heis}^{i,i}.
\end{align}
The Kitaev interaction is 

\begin{align}
    \hat{h}_{\rm Kitaev}^i &= \hat{S}^z_{i,A}\hat{S}^z_{i,B} + \hat{S}^x_{i,A}\hat{S}^x_{i-1,B} + \hat{S}^y_{i,A}\hat{S}^y_{i-2,B}  \notag \\
    &=  \hat{S}^z_{i,A}\hat{S}^z_{i,B} + \frac{\hat{S}^+_{i,A} + \hat{S}^-_{i,A}}{2}\frac{\hat{S}^+_{i-1,B} + \hat{S}^-_{i-1,B}}{2} + \frac{\hat{S}^+_{i,A} - \hat{S}^-_{i,A}}{2i}\frac{\hat{S}^+_{i-2, B} - \hat{S}^-_{i-2, B}}{2i} \notag \\
    &\sim (S-\hat{a}^\dagger_i \hat{a}_i)(\hat{b}^\dagger_i \hat{b}_i -S )\notag   \\ 
    & \quad  +\frac{2S}{4} \left(  \hat{a}_i + \hat{a}^{\dagger}_i \right) \left( \hat{b}^{\dagger}_{i-1} + \hat{b}_{i-1} \right)   -\frac{2S}{4} \left(  \hat{a}_i - \hat{a}^{\dagger}_i \right) \left( \hat{b}^{\dagger}_{i-2} - \hat{b}_{i-2} \right) \notag \\
    &\sim  -S^2 +S \hat{a}^\dagger_i \hat{a}_i  + S\hat{b}^\dagger_i \hat{b}_i  \notag \\ 
    & \quad \quad  + \frac{S}{2} \left( \hat{a}_i\hat{b}_{i-1} + \hat{a}_i\hat{b}_{i-1}^\dagger + \hat{a}_i^\dagger \hat{b}_{i-1} + \hat{a}_i^\dagger\hat{b}_{i-1}^\dagger \right) \notag \\ 
    & \quad \quad  - \frac{S}{2} \left( \hat{a}_i\hat{b}_{i-2}^\dagger - \hat{a}_i\hat{b}_{i-2} - \hat{a}_i^\dagger \hat{b}_{i-2}^\dagger + \hat{a}_i^\dagger\hat{b}_{i-2} \right). 
\end{align} 

We get the Hamiltonian written in HP bosons
\begin{align}
    H &= \cos(2\pi \xi)H_{\rm AKLT} + \sin(2\pi \xi)H_{\rm Kitaev} \notag \\
    &\sim C_{\rm const}(\xi) + C_{\rm AKLT}(\xi)\sum_{i}\left( \hat{h}_{\rm Heis}^{i,i} + \hat{h}_{\rm Heis}^{i,i-1} + \hat{h}_{\rm Heis}^{i,i-2} \right) + C_{\rm Kitaev}(\xi) \sum_{i} \hat{h}_{\rm Kitaev}^i.
\end{align}
Defining the Fourier transformation of HP bosons as eqs.\eqref{eq:FT-HPboson-a} and \eqref{eq:FT-HPboson-b},
Our Hamiltonian with bosons in the reciprocal space is
\begin{align}
    H &= C_{\rm const}(\xi) + C_{\rm AKLT}(\xi)\sum_{\bm{k}} \left(3\hat{a}^\dagger_{\bm{k}}\hat{a}_{\bm{k}} + 3\hat{b}^\dagger_{\bm{k}}\hat{b}_{\bm{k}} + \Gamma^*(\bm{k}) \hat{a}_{\bm{k}}\hat{b}_{-\bm{k}} + \Gamma(\bm{k}) \hat{a}_{\bm{k}}^\dagger\hat{b}_{-\bm{k}}^\dagger\right) \notag \\
    &\quad + C_{\rm Kitaev}(\xi)\sum_{\bm{k}} \left( \hat{a}^\dagger_{\bm{k}}\hat{a}_{\bm{k}} + \hat{b}^\dagger_{\bm{k}}\hat{b}_{\bm{k}} + g^*(\bm{k}) \hat{a}_{\bm{k}}\hat{b}_{-\bm{k}} + g(\bm{k}) \hat{a}_{\bm{k}}^\dagger \hat{b}_{-\bm{k}}^\dagger + f(\bm{k}) \hat{a}_{\bm{k}}^\dagger\hat{b}_{\bm{k}} + f^*(\bm{k}) \hat{a}_{\bm{k}}\hat{b}_{\bm{k}}^\dagger \right),
\end{align}
where
\begin{equation}
    f(\bm{k}) = \frac{1}{2}\left( e^{-i\bm{k}\cdot\bm{a}_1} - e^{-i\bm{k}\cdot\bm{a}_2} \right), \quad g(\bm{k}) = \frac{1}{2}\left( e^{-i\bm{k}\cdot\bm{a}_1} + e^{-i\bm{k}\cdot\bm{a}_2} \right),
\end{equation}
and
\begin{align}
    &C_{\rm const}(\xi) = 3S^2\cos(2\pi\xi)\left(-1+\frac{116}{243}S^2-\frac{16}{243}S^4\right) + (-S^2)\sin(2\pi\xi), \\
    &C_{\rm AKLT}(\xi) = S\cos(2\pi\xi)\left(1-\frac{232}{243}S^2+\frac{16}{81}S^4\right), \\
    &C_{\rm Kitaev}(\xi) = S\sin(2\pi\xi).
\end{align}
Using the basis in eq.\eqref{eq:BdG-form}, we get the $4\times4$ matrix representation of our Hamiltonian.
\begin{tcolorbox}
    \textbf{Neel SU(2) magnon Hamiltonian}

    \begin{equation}
        \mathcal{H} = \frac{1}{2}\begin{pmatrix}
            3C_{\rm A}+C_{\rm K} & C_{\rm K}f & 0 & C_{\rm A}\Gamma + C_{\rm K}g \\
            C_{\rm K}f^* & 3C_{\rm A}+C_{\rm K} & C_{\rm A}\Gamma^* + C_{\rm K}g^*  & 0 \\
            0 & C_{\rm A}\Gamma + C_{\rm K}g & 3C_{\rm A}+C_{\rm K} & C_{\rm K}f^* \\
            C_{\rm A}\Gamma^* + C_{\rm K}g^* & 0 & C_{\rm K}f & 3C_{\rm A}+C_{\rm K} 
        \end{pmatrix}.
    \end{equation}
    As we did in FM case, it can be simplified by introducing 2 by 2 matrices,
    \begin{equation}
        \mathcal{H} = \begin{pmatrix}
            h & \Delta \\
            \Delta & h
        \end{pmatrix},
    \end{equation}
    \begin{equation}
        h = \frac{1}{2}\begin{pmatrix}
            3C_{\rm A}+C_{\rm K} & C_{\rm K}f \\ C_{\rm K}f^* & 3C_{\rm A}+C_{\rm K}
        \end{pmatrix}, \quad \Delta = \frac{1}{2}\begin{pmatrix}
            0 & C_{\rm A}\Gamma + C_{\rm K}g \\ C_{\rm A}\Gamma^* + C_{\rm K}g^* & 0
        \end{pmatrix}.
    \end{equation}
\end{tcolorbox}

\subsection{Zigzag phase}
The magnetic unit cell of the zigzag state has 4 sublattices. 
The zigzag state is the state that the 1D spin up/down chain are stacking alternatively.
We assume that each 1D spin chain is pointing $z$-up or down, and the inter-chain bonds are $z$ directional.
The procedure to get the BdG Hamiltonian is the same as the FM and Neel cases. 
However, as the magnetic unit cell has four sublattices, we have to introduce four HP bosons and 
the BdG Hamiltonian is $8\times8$ matrix.
It is straightforward to show the following result.

\begin{tcolorbox}
    \textbf{zigzag SU(2) magnon Hamiltonian}

    \begin{equation}
        \mathcal{H} = \frac{1}{2}\begin{pmatrix}
            h & \Delta \\ \Delta & h
        \end{pmatrix},
    \end{equation}
    \begin{equation}
        h = \begin{pmatrix}
            g & O \\ O & g
        \end{pmatrix}, g = \begin{pmatrix}
            2A_A + A_F + K & g(K-2A_F) \\ g^*(K-2A_F) & 2A_A + A_F + K 
        \end{pmatrix}
    \end{equation}
    \begin{equation}
        \Delta = \begin{pmatrix}
            a & b \\ b^\dagger & a
        \end{pmatrix}, a = \begin{pmatrix}
            0 & Kf \\ Kf^* & 0 
        \end{pmatrix}, b = \begin{pmatrix}
            0 & A_Ae^{+i\bm{k}\cdot\bm{a}_2} \\ A_Ae^{-i\bm{k}\cdot(\bm{a}_1 + \bm{a}_2)}
        \end{pmatrix}
    \end{equation}
    with
    \begin{equation}
        g = g(\bm{k}) = \frac{1}{2}(1 + e^{+i\bm{k}\cdot\bm{a}_2}), f = f(\bm{k}) = \frac{1}{2}(1 - e^{+i\bm{k}\cdot\bm{a}_2}),
    \end{equation}
    \begin{align}
        A_F &= -S\cos(2\pi\xi)\left(1 + \frac{232}{243}S^2 + \frac{16}{81}S^4\right), \\
        A_A &= S\cos(2\pi\xi)\left(1 - \frac{232}{243}S^2 + \frac{16}{81}S^4\right), \\
        K &= S\sin(2\pi\xi).
    \end{align}


\end{tcolorbox}

\newpage

\section{SU(N) Magnon}
Generalized spin wave theory (GSWT) is formulated in \cite{10.1093/ptep/ptu109}. This paper also discusses the application of the GSWT to the bilinear biquadratic (BBQ) model. 

\subsection{Generalized spin wave theory}
We briefly introduce the generalized spin wave theory in this subsection.
When we consider SU($N$) fluctuation, we prepare $N$ kinds of schwinger bosons which satisfy the following constraint
\begin{equation}\label{eq:GSWT1-1}
    \sum_{i=0}^{N-1} b_{\bm{r} i}^\dagger b_{\bm{r} i} = N\mathcal{S}.
\end{equation}
$N\mathcal{S}$ depends on spin-$S$. For example, if you consider spin-$3/2$ and SU(4), then $N\mathcal{S}=1$
since the fundamental representation of SU(4) has dimension $4=2S+1$.
The SU($N$) generators are expressed by bilinear forms of Schwinger bosons,
\begin{equation}\label{eq:GSWT1-2}
    O^{mm^\prime} = b_{\bm{r} m}^\dagger b_{\bm{r} m^\prime},
\end{equation}
and local operators are represented by the linear combinations of these generators
\begin{equation}\label{eq:GSWT1-3}
    X_{\bm{r}} = \sum_{m,m^\prime}\mathcal{X}_{\bm{r}}^{mm^\prime} \mathcal{O}_{\bm{r}}^{mm^\prime} = \mathbf{b}_{\bm{r}}^\dagger \mathcal{X}_{\bm{r}}\mathbf{b}_{\bm{r}}.
\end{equation}
When we choose several ansatzes, we need to apply unitary transformation to the Schwinger bosons $\tilde{\mathbf{b}_0} = U\mathbf{b}_0$.
The $m=0$ boson is the one to be condensed
\begin{equation}\label{eq:GSWT1-4}
    \tilde{b}_{\bm{r}0}^\dagger = \tilde{b}_{\bm{r}0} = \sqrt{N\mathcal{S}} \sqrt{1-\frac{1}{N\mathcal{S}}\sum_{m=1}^{N-1}\tilde{b}_{\bm{r}m}^\dagger\tilde{b}_{\bm{r}m}}.
\end{equation}

Let us calculate the general interaction by the Schwinger Bosons and perform the Holstein-Prinakoff transformation.
Here we assume $N\mathcal{S}=1$ and the following type of the two-site interaction
\begin{equation}\label{eq:GSWT1-5}
    \hat{A}_{\bm{r}}\hat{B}_{\bm{r^\prime}},
\end{equation}
where $\hat{A}$ and $\hat{B}$ are elements of the SU($N$) algebra.
In the language of the Schwinger Bosons, the interaction \eqref{eq:GSWT1-5} is written as 
\begin{align}\label{eq:GSWT1-6}
    (\mathbf{b}_{\bm{r}}^\dagger A\mathbf{b}_{\bm{r}})(\mathbf{b}_{\bm{r}^\prime}^\dagger B\mathbf{b}_{\bm{r}^\prime}) &= (b_{\bm{r}m}^\dagger A_{mm^\prime}b_{\bm{r}m^\prime})(b_{\bm{r}^\prime n}^\dagger B_{nn^\prime}b_{\bm{r}^\prime n^\prime}).
\end{align}
As the first step, we focus on the $b_{\bm{r}m}^\dagger A_{mm^\prime}b_{\bm{r}m^\prime}$ term. 
When $m=m^\prime=0$, this term is expanded as
\begin{align}\label{eq:GSWT1-7}
    b_{\bm{r}0}^\dagger A_{00}b_{\bm{r}0} &= \sqrt{1-\sum_{i=1}^{N-1}b_{\bm{r}i}^\dagger b_{\bm{r}i}} \ A_{00}  \sqrt{1-\sum_{j=1}^{N-1}b_{\bm{r}j}^\dagger b_{\bm{r}j}} \notag \\
    &\sim \left( 1-\frac{1}{2}\sum_{i=1}^{N-1}b_{\bm{r}i}^\dagger b_{\bm{r}i} \right) A_{00} \left( 1-\frac{1}{2}\sum_{j=1}^{N-1}b_{\bm{r}j}^\dagger b_{\bm{r}j} \right) \notag \\ 
    &\sim A_{00} - A_{00}\sum_{i=1}^{N-1} b_{\bm{r}i}^\dagger b_{\bm{r}i} .
\end{align}
Since we are only interested in the bilinear terms, we ignored the two-body or higher terms.
The $m=0, m^\prime \neq 0$ case is also expanded at the same manner;
\begin{align}\label{eq:GSWT1-8}
    b_{\bm{r}0}^\dagger A_{0m^\prime}b_{\bm{r}m^\prime} &= \sqrt{1-\sum_{i=1}^{N-1}b_{\bm{r}i}^\dagger b_{\bm{r}i}} \ A_{0m^\prime}b_{\bm{r}m^\prime} \notag \\
    &\sim \left( 1-\frac{1}{2}\sum_{i=1}^{N-1}b_{\bm{r}i}^\dagger b_{\bm{r}i} \right) A_{0m^\prime}b_{\bm{r}m^\prime} \notag \\ 
    &=A_{0m^\prime}b_{\bm{r}m^\prime} -\frac{1}{2}\sum_{i=1}^{N-1} A_{0m^\prime} b_{\bm{r}i}^\dagger b_{\bm{r}i}b_{\bm{r}m^\prime} .
\end{align}
The $m=0, m^\prime \neq 0$ term yields one, three, and higher creation/annihilation terms. 
The last case is $m\neq 0, m^\prime \neq 0$ term. These terms do not include $b_0$, so they are simply written as
\begin{align}\label{eq:GSWT1-9}
    b_{\bm{r}m}^\dagger A_{mm^\prime}b_{\bm{r}m^\prime}.
\end{align}
We are interested in constant and bilinear terms. From the observations above, the interaction \eqref{eq:GSWT1-5} is expanded as
\begin{align}\label{eq:GSWT1-10}
    (\mathbf{b}_{\bm{r}}^\dagger A\mathbf{b}_{\bm{r}})(\mathbf{b}_{\bm{r}^\prime}^\dagger B\mathbf{b}_{\bm{r}^\prime}) &= (b_{\bm{r}m}^\dagger A_{mm^\prime}b_{\bm{r}m^\prime})(b_{\bm{r}^\prime n}^\dagger B_{nn^\prime}b_{\bm{r}^\prime n^\prime}) \notag \\
    &= A_{00}B_{00} \notag \\
    &+ A_{00} \sum_{i=1}^{N-1}\left( B_{i0}b_{\bm{r}^\prime i}^\dagger +  B_{0i}b_{\bm{r}^\prime i}\right) + B_{00} \sum_{i=1}^{N-1}\left( A_{i0}b_{\bm{r} i}^\dagger +  A_{0i}b_{\bm{r} i}\right) \notag \\
    &+ \left[ \sum_{i=1}^{N-1}\left( A_{i0}b_{\bm{r} i}^\dagger +  A_{0i}b_{\bm{r} i}\right) \right] \left[ \sum_{i=1}^{N-1}\left( B_{i0}b_{\bm{r}^\prime i}^\dagger +  B_{0i}b_{\bm{r}^\prime i}\right) \right] \notag \\
    &- A_{00}B_{00}\sum_{i=1}^{N-1} b_{\bm{r}i}^\dagger b_{\bm{r}i} - A_{00}B_{00}\sum_{i=1}^{N-1} b_{\bm{r}^\prime i}^\dagger b_{\bm{r}^\prime i} \notag \\
    &+ A_{00}\sum_{n,n^\prime=1}^{N-1}B_{nn^\prime} b_{\bm{r}^\prime n}^\dagger b_{\bm{r}^\prime n^\prime} + B_{00}\sum_{m,m^\prime=1}^{N-1}A_{mm^\prime}b_{\bm{r}m}^\dagger b_{\bm{r}m^\prime}.
\end{align}
The first line is constant term, and the second line is one operator terms, and from the third to the last line are the two operator (bilinear) terms.
It is convenient to introduce the following quantity:
\begin{equation}\label{eq:GSWT1-11}
    t^{\alpha\beta}_{\gamma\delta} = A_{\alpha\beta}B_{\gamma\delta}.
\end{equation}
Then, the bilinear terms in \eqref{eq:GSWT1-10} are
\begin{align}\label{eq:GSWT1-12}
    (\mathbf{b}_{\bm{r}}^\dagger A\mathbf{b}_{\bm{r}})(\mathbf{b}_{\bm{r}^\prime}^\dagger B\mathbf{b}_{\bm{r}^\prime}) &= \sum_{m,n=1}^{N-1} \left( t^{m0}_{n0}b_{\bm{r}m}^\dagger b_{\bm{r}^\prime n}^\dagger +  t^{m0}_{0n}b_{\bm{r}m}^\dagger b_{\bm{r}^\prime n} + t^{0m}_{n0}b_{\bm{r}m} b_{\bm{r}^\prime n}^\dagger + t^{0m}_{0n}b_{\bm{r}m} b_{\bm{r}^\prime n} \right) \notag \\
    &+ \sum_{m,n=1}^{N-1} \left( t^{mn}_{00}b_{\bm{r}m}^\dagger b_{\bm{r}n} + t^{00}_{mn}b_{\bm{r}^\prime m}^\dagger b_{\bm{r}^\prime n} \right) - t^{00}_{00} \sum_{i=1}^{N-1} \left( b_{\bm{r}i}^\dagger b_{\bm{r}i} + b_{\bm{r}^\prime i}^\dagger b_{\bm{r}^\prime i}\right)
\end{align}
As we did in the SU(2) magnon, it is convenient to perform the Fourier transformation.
We define the Fourier transformation of the Schwinger Bosons as 
\begin{equation}\label{eq:GSWT1-13}
    b_{\bm{k}m}^\dagger = \frac{1}{\sqrt{N}} \sum_{i}b_{im}^\dagger e^{i\bm{k}\cdot \bm{R}_i},
\end{equation}
where $i$ is the unit cell index and $\bm{R}_i$ denotes the position of the $i$-th unit cell.
When we assume our system is on the honeycomb lattice and our Hamiltonian has only nearest neighbor interactions, 
eq.\eqref{eq:GSWT1-12} becomes
\begin{align}\label{eq:GSWT1-14}
    \sum_{\langle \bm{r}, \bm{r}^\prime \rangle}(\mathbf{a}_{\bm{r}}^\dagger A\mathbf{a}_{\bm{r}})(\mathbf{b}_{\bm{r}^\prime}^\dagger B\mathbf{b}_{\bm{r}^\prime}) &= \sum_{\bm{k}}\sum_{m,n=1}^{N-1}(  t^{m0}_{n0} \Gamma^*(\bm{k}) a_{\bm{k},m}^\dagger b_{-\bm{k},n}^\dagger + t^{m0}_{0n} \Gamma(\bm{k}) a_{\bm{k},m}^\dagger b_{\bm{k},n}  \notag \\
    &+ t^{0m}_{n0} \Gamma^*(\bm{k})  b_{\bm{k},n}^\dagger a_{\bm{k},m} + t^{0m}_{0n} \Gamma(\bm{k}) a_{\bm{k},m}  b_{-\bm{k},n}  ) \notag \\
    &+ \sum_{\bm{k}}\sum_{m,n=1}^{N-1} (3t^{mn}_{00}a_{\bm{k}m}^\dagger a_{\bm{k}n} + 3t^{00}_{mn}b_{\bm{k}m}^\dagger b_{\bm{k}n}) \notag \\
    &- 3t^{00}_{00} \sum_{i=1}^{N-1} \left( a_{\bm{k}i}^\dagger a_{\bm{k}i} +b_{\bm{k}i}^\dagger b_{\bm{k}i} \right),
\end{align}
where 
\begin{equation}
    \Gamma(\bm{k}) = 1 + e^{-i\bm{k}\cdot \bm{R}_1} + e^{-i\bm{k}\cdot \bm{R}_2}.
\end{equation}

\subsection{FM phase in the Kitaev-AKLT model}
In this subsection we derive the SU($N$) magnon Hamiltonian by assuming the Ferro magnetic state under the Kitaev-AKLT Hamiltonian.
As for the AKLT interaction, We only have to utilize the eq. \eqref{eq:GSWT1-14} by setting
\begin{equation}
    (t_{\rm AKLT})^{\alpha\beta}_{\gamma\delta}  \equiv A^{\alpha\beta}_{\gamma\delta} = \sum_{\mu\in \left\{ x,y,z \right\}} S^\mu_{\alpha\beta}S^\mu_{\gamma\delta} + \frac{116}{243}\sum_{\mu,\nu \in \left\{ x,y,z \right\}} S^{\mu\nu}_{\alpha\beta}S^{\mu\nu}_{\gamma\delta}  + \frac{16}{243}\sum_{\mu,\nu,\rho \in \left\{ x,y,z \right\}} S^{\mu\nu\rho}_{\alpha\beta}S^{\mu\nu\rho}_{\gamma\delta},
\end{equation}
with $S^{\mu\nu\rho} = S^{\mu}S^{\nu}S^{\rho}$.
The Kitaev interaction is slightly trickey.
Defining two quantities
\begin{align}
    &K^{\alpha\beta}_{\gamma\delta} =  \sum_{\mu\in \left\{ x,y,z \right\}}S^\mu_{\alpha\beta}S^\mu_{\gamma\delta}, \\
    &\tilde{K}^{\alpha\beta}_{\gamma\delta} = S^z_{\alpha\beta}S^z_{\gamma\delta} + S^x_{\alpha\beta}S^x_{\gamma\delta}e^{-i\bm{k}\cdot \bm{R}_1}  + S^y_{\alpha\beta}S^y_{\gamma\delta}e^{-i\bm{k}\cdot \bm{R}_2},
\end{align}
The Kitaev interaction term is transformed as
\begin{align}
    \sum_{i} S^z_{iA}S^z_{iB} + S^x_{iA}S^x_{(i-1)B}+ S^y_{iA}S^y_{(i-2)B}  &= \sum_{\bm{k}}\sum_{m,n=1}^{N-1}(  (\tilde{K}^{m0}_{n0})^* a_{\bm{k},m}^\dagger b_{-\bm{k},n}^\dagger + \tilde{K}^{m0}_{0n} a_{\bm{k},m}^\dagger b_{\bm{k},n}  \notag \\
    &+ (\tilde{K}^{0m}_{n0})^* b_{\bm{k},n}^\dagger a_{\bm{k},m} + \tilde{K}^{0m}_{0n} a_{\bm{k},m}  b_{-\bm{k},n}  ) \notag \\
    &+ \sum_{\bm{k}}\sum_{m,n=1}^{N-1} (K^{mn}_{00}a_{\bm{k}m}^\dagger a_{\bm{k}n} + K^{00}_{mn}b_{\bm{k}m}^\dagger b_{\bm{k}n}) \notag \\
    &- K^{00}_{00} \sum_{i=1}^{N-1} \left( a_{\bm{k}i}^\dagger a_{\bm{k}i} +b_{\bm{k}i}^\dagger b_{\bm{k}i} \right).
\end{align}
Our Hamiltonian is given by \eqref{eq:Hamiltonian}. Modifying $A,K,\tilde{K}$ as 
\begin{align}
    &A^{\alpha\beta}_{\gamma\delta} = \cos (2\pi\xi) \left( \sum_{\mu\in \left\{ x,y,z \right\}} S^\mu_{\alpha\beta}S^\mu_{\gamma\delta} + \frac{116}{243}\sum_{\mu,\nu \in \left\{ x,y,z \right\}} S^{\mu\nu}_{\alpha\beta}S^{\mu\nu}_{\gamma\delta}  + \frac{16}{243}\sum_{\mu,\nu,\rho \in \left\{ x,y,z \right\}} S^{\mu\nu\rho}_{\alpha\beta}S^{\mu\nu\rho}_{\gamma\delta} \right), \\
    &K^{\alpha\beta}_{\gamma\delta} = \sin(2\pi \xi) \sum_{\mu\in \left\{ x,y,z \right\}}S^\mu_{\alpha\beta}S^\mu_{\gamma\delta}, \\
    &\tilde{K}^{\alpha\beta}_{\gamma\delta} = \sin(2\pi \xi) \left( S^z_{\alpha\beta}S^z_{\gamma\delta} + S^x_{\alpha\beta}S^x_{\gamma\delta}e^{-i\bm{k}\cdot \bm{R}_1}  + S^y_{\alpha\beta}S^y_{\gamma\delta}e^{-i\bm{k}\cdot \bm{R}_2} \right) ,
\end{align}
the Kitaev-AKLT Hamiltonian becomes
\begin{align}
    H &= \sum_{\bm{k}}\sum_{m,n=1}^{N-1}( \left[A^{m0}_{n0}\Gamma^*(\bm{k}) + (\tilde{K}^{m0}_{n0})^*  \right] a_{\bm{k},m}^\dagger b_{-\bm{k},n}^\dagger +  \left[A^{m0}_{0n}\Gamma(\bm{k}) + \tilde{K}^{m0}_{0n} \right] a_{\bm{k},m}^\dagger b_{\bm{k},n}  \notag \\
    &+ \left[A^{0m}_{n0}\Gamma^*(\bm{k}) + (\tilde{K}^{0m}_{n0})^*  \right]  b_{\bm{k},n}^\dagger a_{\bm{k},m} + \left[A^{0m}_{0n}\Gamma(\bm{k}) + \tilde{K}^{0m}_{0n}  \right] a_{\bm{k},m}  b_{-\bm{k},n}  ) \notag \\
    &+ \sum_{\bm{k}}\sum_{m,n=1}^{N-1} (\left[3A^{mn}_{00} + {K}^{mn}_{00} \right]a_{\bm{k}m}^\dagger a_{\bm{k}n} + \left[3A^{00}_{mn} + {K}^{00}_{mn} \right]b_{\bm{k}m}^\dagger b_{\bm{k}n}) \notag \\
    &- \left[3A^{00}_{00} + {K}^{00}_{00} \right] \sum_{i=1}^{N-1} \left( a_{\bm{k}i}^\dagger a_{\bm{k}i} +b_{\bm{k}i}^\dagger b_{\bm{k}i} \right).
\end{align}

\begin{tcolorbox}
    In the matrix form, setting basis as $\left( \mathbf{a}_{\bm{k}}^\dagger, \mathbf{b}_{\bm{k}}^\dagger, \mathbf{a}_{-\bm{k}}, \mathbf{b}_{-\bm{k}} \right)$, we get 
\begin{equation}
    \mathcal{H} = \frac{1}{2}\begin{pmatrix}
        h & \Delta \\ \Delta & h
    \end{pmatrix},
\end{equation}

\begin{equation}
    h = \begin{pmatrix}
        \left( 3A+K \right)^{mn}_{00}-\delta_{mn}\left( 3A+K \right)^{00}_{00} & A^{m0}_{0n}\Gamma + \tilde{K}^{m0}_{0n} \\
        A^{0m}_{n0}\Gamma^* + (\tilde{K}^{0m}_{n0})^* & \left(   3A+K \right)^{00}_{mn}-\delta_{mn}\left( 3A+K \right)^{00}_{00}
    \end{pmatrix},
\end{equation}

\begin{equation}
    \Delta = \begin{pmatrix}
        0 & A^{0m}_{0n}\Gamma + \tilde{K}^{0m}_{0n} \\ [A^{m0}_{n0}\Gamma^* + \tilde{K}^{m0}_{n0}]^T & 0
    \end{pmatrix}.
\end{equation}
\end{tcolorbox}

\newpage

\section*{update history}
\begin{itemize}
    \item 2025/1/11 First draft by Sogen Ikegami
    \item 2025/1/15 Add Section \ref{sec:equivalence} 
    \item 2025/2/20 Add some thought about classical spin length in Sec.\ref{sec:equivalence} 
    \item 2025/2/26 Add SU(2) magnon Section \ref{sec:SU2_magnon}
    \item 2025/3/17 Add SU(N) magnon Section
    \item 2025/4/15 Revise SU(2) magnon Hamiltonian. $\sum_{\bm{k}}e^{i\bm{k}\cdot \bm{a}}\hat{a}^\dagger_{\bm{k}} \hat{a}_{\bm{k}} = \sum_{\bm{k}}e^{\red{-}i\bm{k}\cdot \bm{a}}\hat{a}^\dagger_{\bm{-k}} \hat{a}_{\bm{-k}}$. Originally I missed the minus sign in the exponential factor in the r.h.s.
        Correspondingly I revised the BdG Hamiltonian.
\end{itemize}

\newpage 

\printbibliography
%\bibliographystyle{plain}
%\bibliography{reference.bib}



\end{document}